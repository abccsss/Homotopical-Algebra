In infinity categories,
homotopy equivalent objects are equivalent by definition.
Therefore, the only well-defined notion of (co)limits 
is that of homotopy (co)limits.
It is very difficult to compute homotopy (co)limits
directly from the definitions.
However, model categories will give us great help in such computations.

\subsection{Colimits and limits}

The simplest colimit is the empty colimit, that is, the initial object.
In ordinary category theory, the initial object is characterised by
the property that it admits a unique morphism to any other object.

The $\infty$-categorical way of saying something is unique 
is to say that all possible choices form a contractible space,
i.e.\ a contractible Kan complex.

\begin{definition}
    Let $\cat C$ be a quasi-category.
    An object $x\in\cat C$ is called an \term{initial object},
    if for any $y\in\cat C$, the mapping space $\Hom_{\cat C}(x,y)$ is contractible.
\end{definition}

The notation $\Hom_{\cat C}(x,y)$ refers to any one of
\[\Hom_{\cat C}^\vartriangleleft(x,y),\quad
\Hom_{\cat C}^\vartriangleright(x,y),\quad 
\Hom_{\mathfrak C\cat C}(x,y),\quad\text{etc.,} \]
which all have the same homotopy type.

\begin{remark}
    An initial object of $\cat C$
    is equivalently an initial object of the $\cat{hCW}$-enriched category $h\cat C$.
    In fact, most of the notions defined in this section will be equivalent to 
    the corresponding notions for $\cat{hCW}$-enriched categories. \varqed
\end{remark}

One should expect that initial objects are unique if they exist.
As before, uniqueness means being a contractible space.

\begin{proposition}
    Let $\cat C$ be a quasi-category.
    \begin{itms}
        \item An object $x\in\cat C$ is an initial object, if and only if
        the left fibration
        \[ \cat C_{x\/}\to\cat C \]
        is a trivial fibration.
        \item The full subcategory spanned by the initial objects 
        is either empty, or a contractible Kan complex.
    \end{itms}
\end{proposition}

\begin{proof}
    \nyw
\end{proof}

Colimits are nothing but initial objects of under-categories.

\begin{definition}
    Let $\cat C$ be a quasi-category,
    $K$ a simplicial set,
    and let $f\:K\to\cat C$ be a diagram.
    A \term{colimit} of $f$ is an initial object of $\cat C_{K\/}$.
\end{definition}

We immediately deduce the following.

\begin{corollary}
    Let $\cat C$ be a quasi-category,
    $K$ a simplicial set,
    and let $f\:K\to\cat C$ be a diagram.
    \begin{itms}
        \item A map $\bar f\:K^\vartriangleright\to\cat C$
        is a colimit of $f$, if and only if
        the induced left fibration
        \[ \cat C_{K^\vartriangleright\/}\to\cat C_{K\/} \]
        is a trivial fibration.
        \item The category of colimits of $f$ 
        is either empty, or a contractible Kan complex.
    \end{itms}
\end{corollary}

\begin{proof}
    We have an isomorphism of simplicial sets
    \[ \cat C_{K^\vartriangleright\/}\simeq(\cat C_{K\/})_{x\/}, \]
    where $x$ denotes the image of the cone point of $K^\vartriangleright$.
    Everything else is clear.
\end{proof}

\begin{remark}\label{thm-6-c}
    The natural map 
    \[ \cat C_{K^\vartriangleright\/}\to\cat C_{x\/} \]
    is always a trivial fibration,
    as can be shown by verifying the lifting property,
    by transfinite induction on the number of simplices of $K$.
    Details are left to the reader. \varqed
\end{remark}

Colimits are not computable via this definition.
We need to deduce some of their properties to make them computable.

\begin{proposition}
    Let $\cat C$ be a quasi-category,
    and let $\{x_\alpha\}$ be a collection of objects in $\cat C$.
    An object $x\in\cat C$ is a coproduct of the objects $x_\alpha$,
    if and only if for any $y\in\cat C$, the induced map 
    \[ \Hom_{\cat C}(x,y)\to\prod_\alpha\Hom_{\cat C}(x_\alpha,y) \]
    is a homotopy equivalence.
\end{proposition}

\begin{proof}
    The right hand side is equivalent to  
    \[ \prod_\alpha{}(\cat C_{x_\alpha\/})_y
    \simeq (\cat C_{\{x_\alpha\}\/})_y, \]
    where the subscript $y$ means taking the fibre of the map to $\cat C$.
    The left hand side is equivalent to 
    \[ (\cat C_{x\/})_y\simeq((\cat C_{\{x_\alpha\}\/})_{x\/})_y \]
    by (\ref{thm-6-c}). Thus $x$ is a coproduct, if and only if the left fibration
    \[(\cat C_{\{x_\alpha\}\/})_{x\/}\to\cat C_{\{x_\alpha\}\/}\]
    is a trivial fibration,
    if and only if it is a DK equivalence (since DK equivalences are weak homotopy equivalences),
    if and only if their fibres are equivalent, by the next lemma.
\end{proof}

\begin{lemma}
    Let $S$ be a simplicial set, and let $X,Y\in\cat{sSet}_{\/S}$
    be two left fibrations over $S$.
    Let $f\:X\to Y$ be a map in $\cat{sSet}_{\/S}$.
    Then $f$ is a DK equivalence if and only if 
    $f$ induces weak equivalences on each fibre.
\end{lemma}

\begin{proof}
    \nyw
\end{proof}

This result on coproducts is a special case of a general theorem,
which we state below.

Let $\cat K$ be a category enriched over Kan complexes.
We denote by $\cat K^\vartriangleright$ the $\cat{Kan}$-enriched category obtained
by adjoining a terminal object $\infty$, with 
\[ \Hom_{\cat K^\vartriangleright}(x,\infty)=\{*\}\quad(x\in\cat K^\vartriangleright),\quad
\Hom_{\cat K^\vartriangleright}(\infty,x)=\emptyset\quad(x\in\cat K).\quad  \]

\begin{definition}
    Let $\cat C$ and $\cat K$ be categories enriched over Kan complexes,
    and let $f\:\cat K\to\cat C$ be a functor, which we see as a diagram in $\cat C$.
    A \term{homotopy colimit} of $f$ is a functor 
    \[ \bar f\:\cat K^\vartriangleright\to\cat C, \]
    such that $\bar f|_{\cat K}=f$, and for any $y\in\cat C$, the induced map 
    \[ \Hom_{\cat C}(x,y)\to\mathop{\operatorname{holim}}\limits_{k\in\cat K}\Hom_{\cat C}(f(k),y) \]
    is a homotopy equivalence of Kan complexes,
    where $x:=\bar f(\infty)$, and $\operatorname{holim}$ denotes the homotopy limit of Kan complexes.
\end{definition}

\tbc

\subsection{Adjoint functors}

\nyw

