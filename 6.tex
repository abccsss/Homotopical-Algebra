In infinity categories,
homotopy equivalent objects are equivalent by definition.
Therefore, the only well-defined notion of (co)limits 
is that of homotopy (co)limits.
It is very difficult to compute homotopy (co)limits
directly from the definitions.
However, model categories will give us great help in such computations.

\subsection{Colimits and limits}

The simplest colimit is the empty colimit, that is, the initial object.
In ordinary category theory, the initial object is characterised by
the property that it admits a unique morphism to any other object.

The $\infty$-categorical way of saying something is unique 
is to say that all possible choices form a contractible space,
i.e.\ a contractible Kan complex.

\begin{definition}
    Let $\cat C$ be a quasi-category.
    An object $x\in\cat C$ is called an \term{initial object},
    if for any $y\in\cat C$, the mapping space $\Hom_{\cat C}(x,y)$ is contractible.
\end{definition}

The notation $\Hom_{\cat C}(x,y)$ refers to any one of
\[\Hom_{\cat C}^\vartriangleleft(x,y),\quad
\Hom_{\cat C}^\vartriangleright(x,y),\quad 
\Hom_{\mathfrak C\cat C}(x,y),\quad\text{etc.,} \]
which all have the same homotopy type.

\begin{remark}
    An initial object of $\cat C$
    is equivalently an initial object of the $\cat{hCW}$-enriched category $h\cat C$.
    In fact, most of the notions defined in this section will be equivalent to 
    the corresponding notions for $\cat{hCW}$-enriched categories. \varqed
\end{remark}

One should expect that initial objects are unique if they exist.
As before, uniqueness means being a contractible space.

\begin{proposition}
    Let $\cat C$ be a quasi-category.
    \begin{itms}
        \item An object $x\in\cat C$ is an initial object, if and only if
        the left fibration
        \[ \cat C_{x\/}\to\cat C \]
        is a trivial fibration.
        \item The full subcategory spanned by the initial objects 
        is either empty, or a contractible Kan complex.
    \end{itms}
\end{proposition}

\begin{proof}
    \nyw
\end{proof}

Colimits are nothing but initial objects of under-categories.

\begin{definition}
    Let $\cat C$ be a quasi-category,
    $K$ a simplicial set,
    and let $f\:K\to\cat C$ be a diagram.
    A \term{colimit} of $f$ is an initial object of $\cat C_{K\/}$.
\end{definition}

We immediately deduce the following.

\begin{corollary}
    Let $\cat C$ be a quasi-category,
    $K$ a simplicial set,
    and let $f\:K\to\cat C$ be a diagram.
    \begin{itms}
        \item A map $\bar f\:K^\vartriangleright\to\cat C$
        is a colimit of $f$, if and only if
        the induced left fibration
        \[ \cat C_{K^\vartriangleright\/}\to\cat C_{K\/} \]
        is a trivial fibration.
        \item The category of colimits of $f$ 
        is either empty, or a contractible Kan complex.
    \end{itms}
\end{corollary}

\begin{proof}
    We have an isomorphism of simplicial sets
    \[ \cat C_{K^\vartriangleright\/}\simeq(\cat C_{K\/})_{x\/}, \]
    where $x$ denotes the image of the cone point of $K^\vartriangleright$.
    Everything else is clear.
\end{proof}

\begin{remark}\label{thm-6-c}
    The natural map 
    \[ \cat C_{K^\vartriangleright\/}\to\cat C_{x\/} \]
    is always a trivial fibration,
    as can be shown by verifying the lifting property,
    by transfinite induction on the number of simplices of $K$.
    Details are left to the reader. \varqed
\end{remark}

Colimits are not computable via this definition.
We need to deduce some of their properties to make them computable.

\begin{proposition}
    Let $\cat C$ be a quasi-category,
    and let $\{x_\alpha\}$ be a collection of objects in $\cat C$.
    An object $x\in\cat C$ is a coproduct of the objects $x_\alpha$,
    if and only if for any $y\in\cat C$, the induced map 
    \[ \Hom_{\cat C}(x,y)\to\prod_\alpha\Hom_{\cat C}(x_\alpha,y) \]
    is a homotopy equivalence.
\end{proposition}

\begin{proof}
    The right hand side is equivalent to  
    \[ \prod_\alpha{}(\cat C_{x_\alpha\/})_y
    \simeq (\cat C_{\{x_\alpha\}\/})_y, \]
    where the subscript $y$ means taking the fibre of the map to $\cat C$.
    The left hand side is equivalent to 
    \[ (\cat C_{x\/})_y\simeq((\cat C_{\{x_\alpha\}\/})_{x\/})_y \]
    by (\ref{thm-6-c}). Thus $x$ is a coproduct, if and only if the left fibration
    \[(\cat C_{\{x_\alpha\}\/})_{x\/}\to\cat C_{\{x_\alpha\}\/}\]
    is a trivial fibration,
    if and only if it is a DK equivalence (since DK equivalences are weak homotopy equivalences),
    if and only if their fibres are equivalent, by the next lemma.
\end{proof}

\begin{lemma}
    Let $S$ be a simplicial set, and let $X,Y\in\cat{sSet}_{\/S}$
    be two left fibrations over $S$.
    Let $f\:X\to Y$ be a map in $\cat{sSet}_{\/S}$.
    Then $f$ is a DK equivalence if and only if 
    $f$ induces weak equivalences on each fibre.
\end{lemma}

\begin{proof}
    \nyw
\end{proof}

This result on coproducts is a special case of a general theorem,
which we state below.

Let $\cat K$ be a category enriched over Kan complexes.
Then the category of simplicially enriched functors
\[ \operatorname{Fun}_{\cat{sSet}}(\cat K,\cat{sSet}) \]
carries two model structures:
\begin{itms}
    \item The \term{projective model structure},
    with weak equivalences and fibrations defined pointwise,
    and cofibrations defined by the lifting property.
    \item The \term{injective model structure},
    with weak equivalences and cofibrations defined pointwise,
    and fibrations defined by the lifting property.
\end{itms}
The constant functor
\[ \operatorname{const}\:\cat{sSet}\to\operatorname{Fun}_{\cat{sSet}}(\cat K,\cat{sSet}) \]
is right Quillen with respect to the projective model structure,
and left Quillen with respect to the injective model structure.
Therefore, the colimit functor 
\[ \operatorname{colim}\:\operatorname{Fun}_{\cat{sSet}}(\cat K,\cat{sSet})\to\cat{sSet} \]
is left Quillen with respect to the projective model structure,
as it is the left adjoint of the constant functor.
The limit functor 
\[ \operatorname{lim}\:\operatorname{Fun}_{\cat{sSet}}(\cat K,\cat{sSet})\to\cat{sSet} \]
is right Quillen with respect to the injective model structure,
as it is the right adjoint of the constant functor.

\begin{definition}
    Let $\cat K$ be a category enriched over Kan complexes.
    \begin{itms}
        \item The \term{homotopy colimit} functor
        \[ \operatorname{hocolim}\:\Ho(\operatorname{Fun}_{\cat{sSet}}(\cat K,\cat{sSet}))\to\Ho(\cat{sSet}) \]
        is the left derived functor of the left Quillen functor 
        \[ \operatorname{colim}\:\operatorname{Fun}_{\cat{sSet}}(\cat K,\cat{sSet})\to\cat{sSet}, \]
        where $\operatorname{Fun}_{\cat{sSet}}(\cat K,\cat{sSet})$
        is equipped with the projective model structure.

        \item The \term{homotopy limit} functor
        \[ \operatorname{holim}\:\Ho(\operatorname{Fun}_{\cat{sSet}}(\cat K,\cat{sSet}))\to\Ho(\cat{sSet}) \]
        is the right derived functor of the right Quillen functor 
        \[ \operatorname{lim}\:\operatorname{Fun}_{\cat{sSet}}(\cat K,\cat{sSet})\to\cat{sSet}, \]
        where $\operatorname{Fun}_{\cat{sSet}}(\cat K,\cat{sSet})$
        is equipped with the injective model structure.
    \end{itms}
\end{definition}

\begin{example}
    The homotopy pushout
    \[ \{*\}\underset{\{*,*\}}{\sqcup}\{*\} \]
    is the homotopy type of $S^1$, as is familiar in topology.
    A proof will be given later.
    \varqed
\end{example}

We can now use homotopy limits of simplicial sets 
to define homotopy (co)limits in categories enriched over Kan complexes.

Let $\cat K$ be a category enriched over Kan complexes.
We denote by $\cat K^\vartriangleright$ the $\cat{Kan}$-enriched category obtained
by adjoining a terminal object $\infty$, with 
\[ \Hom_{\cat K^\vartriangleright}(x,\infty)=\{*\}\quad(x\in\cat K^\vartriangleright),\quad
\Hom_{\cat K^\vartriangleright}(\infty,x)=\emptyset\quad(x\in\cat K).\quad  \]
Let $\cat K^\vartriangleleft$ be defined dually.

\begin{definition}
    Let $\cat C$ and $\cat K$ be categories enriched over Kan complexes,
    and let $f\:\cat K\to\cat C$ be a functor.
    \begin{itms}
        \item A \term{homotopy colimit} of $f$ is a functor 
        \[ \bar f\:\cat K^\vartriangleright\to\cat C, \]
        such that $\bar f|_{\cat K}=f$, and for any $y\in\cat C$, the induced map 
        \[ \Hom_{\cat C}(\bar f(\infty),y)\to\mathop{\operatorname{holim}}\limits_{k\in\cat K}\Hom_{\cat C}(f(k),y) \]
        is a weak homotopy equivalence.

        \item A \term{homotopy limit} of $f$ is a functor 
        \[ \bar f\:\cat K^\vartriangleleft\to\cat C, \]
        such that $\bar f|_{\cat K}=f$, and for any $y\in\cat C$, the induced map 
        \[ \Hom_{\cat C}(y,\bar f(\infty))\to\mathop{\operatorname{holim}}\limits_{k\in\cat K}\Hom_{\cat C}(y,f(k)) \]
        is a weak homotopy equivalence.
    \end{itms}
\end{definition}

This definition can be seen as the definition of $\infty$-(co)limits 
in the model of categories enriched over Kan complexes.
The following theorem states that
(co)limits in quasi-categories coincide with homotopy (co)limits in this sense.

\begin{theorem}
    Let $\cat C$ and $\cat K$ be categories enriched over Kan complexes,
    and let $f\:\cat K\to\cat C$ be a functor.
    Then the functor between quasi-categories
    \[ \mathfrak Nf\:\mathfrak N\cat K\to\mathfrak N\cat C \]
    has a (co)limit,
    if and only if $f$ has a homotopy (co)limit,
    and in this case, they coincide.
\end{theorem}

See \cite[Theorem~4.2.4.1]{htt}.

Of course, this also means that if $f\:\cat K\to\cat C$
is a functor of quasi-categories, then its (co)limits are computed by
the homotopy (co)limit of the functor
\[ R\mathfrak Cf\:R\mathfrak C\cat K\to R\mathfrak C\cat C \]
between $\cat{Kan}$-enriched categories.
This is because (co)limits are preserved by DK equivalences,
i.e.\ categorical equivalences.

\begin{example}
    When $\cat K=\Lambda_0[2]$, one has an isomorphism
    $\cat K^\vartriangleright\simeq\Delta[1]\times\Delta[1]$.
    The colimit of a diagram $\cat K\to\cat C$ is called a (homotopy) \term{pushout},
    which is a square in $\cat C$, i.e.\ a map $\Delta[1]\times\Delta[1]\to\cat C$.
    We denote pushouts by the usual notation $y\sqcup_xz$.
    By the theorem, we have 
    \[ \Hom_{\cat C}\bigl(y\underset{x}{\sqcup}z,w\bigr)\simeq
    \Hom_{\cat C}(y,w)\underset{\Hom_{\cat C}(x,w)}{\times}\Hom_{\cat C}(z,w) \]
    for any $w\in\cat C$,
    where the right hand side denotes a homotopy pullback of Kan complexes. 
    \varqed
\end{example}

As a corollary, we have the following criterion
that establishes certain ordinary pushouts as homotopy pushouts.

A \term{simplicial model category}
is a model category together with a simplicial enrichment,
together with some (rather strong) compatibility conditions,
which are satisfied by most examples we know.
We do not give the precise definition here.

\begin{proposition}
    Let $\cat C$ be a simplicial model category.
    The homotopy pushout 
    \[ y\underset{x}{\sqcup}z \]
    coincides with the ordinary pushout,
    if the map $x\to y$ is a cofibration.
\end{proposition}

The proof is a standard argument in model category theory 
which we omit here.

This gives an easy way to compute homotopy pushouts
in a simplicial model category. Namely,
to compute $y\sqcup_xz$, one finds a weak equivalence $y\to y'$,
such that the composition $x\to y\to y'$ is a cofibration.
Then the homotopy pushout $y\sqcup_xz$
is given by the ordinary pushout $y'\sqcup_xz$. For example,
in $\cat{Top}$, one has the homotopy pushout
\[ \{*\}\underset{X}{\sqcup}\{*\}\simeq\Sigma X. \]

\begin{remark}
    For simplicial model categories, 
    the $\infty$-categorical (co)limits coincide with 
    the derived functors of the (co)limit functors,
    provided that the projective and injective model structures exist.
    \varqed
\end{remark}

Finally, we mention a formula that computes homotopy (co)limits directly.
Readers unfamiliar with (co)ends may skip to the next subsection.

\begin{proposition}
    Let $f\:\cat K\to\cat C$ be a functor of $\cat{Kan}$-enriched categories.
    Assume that $\cat C$ is tensored and cotensored over $\cat{sSet}$.
    For example, this is always the case if $\cat C$ is a simplicial model category.
    
    \begin{itms}
        \item The homotopy colimit of $f$
        is computed by the simplicially enriched coend 
        \[ \operatorname{hocolim}f\simeq\int^{k\in\cat K}\mathfrak{N}(\cat K_{k\/})\otimes f(k), \]
        where $\otimes$ denotes the simplicial tensoring on $\cat C$.
        \item The homotopy limit of $f$ is computed by the simplicially enriched end
        \[ \operatorname{holim}f\simeq\int_{k\in\cat K}f(k)^{\mathfrak{N}(\cat K_{\/k})}, \]
        where the powering denotes the simplicial powering on $\cat C$.
    \end{itms}
\end{proposition}

See for example \cite{riehl}.

\begin{example}
    Let $G$ be a topological group acting on a space $X$.
    Then the \term{homotopy quotient} $X\/{}^{\mathrm h}G$ is defined by 
    \[ X\/{}^{\mathrm h}G:=(X\times EG)\/G, \]
    where $EG$ denotes the universal $G$-bundle over $BG$.
    This is exactly the homotopy colimit of the corresponding functor 
    \[ \cat{B}G\to\cat{Top}, \]
    where $\cat BG$ is the simplicial groupoid with a single object,
    whose mapping space is $\operatorname{Sing}G$. \varqed
\end{example}

\subsection{Adjoint functors}

In classical category theory,
adjoint functors have played a very important role.
For example, colimits and limits are special cases of adjoint functors.
For $\infty$-categories,
there is also a homotopical version of adjoint functors,
which we will now study.

\begin{definition}
    Let $\cat C,\cat D$ be quasi-categories, and let 
    \[ F\:\cat C\to\cat D,\quad G\:\cat D\to\cat C \]    
    be two functors. We say that $F$ and $G$ are a pair of 
    \term{adjoint functors} if there is an equivalence of functors 
    \[ \Hom_{\cat D}(F-,-)\simeq\Hom_{\cat C}(-,G-)\:\cat C\op\times\cat D\to\cat S. \]
    Precisely, we mean that the functors
    \[ \Hom_{\mathfrak C\cat D}(F-,-)\quad\text{and}\quad\Hom_{\mathfrak C\cat C}(-,G-) \]
    are weakly equivalent in the category 
    \[ \operatorname{Fun}_{\cat{sSet}}(\mathfrak C(\cat C\op\times\cat D),\cat{sSet}), \]
    equipped with the projective model structure.
\end{definition}

In order to understand the properties of adjoint functors,
we will formulate another two equivalent definitions.

\begin{definition}
    Let $\cat C$ be a quasi-category.
    
    \begin{itms}
        \item The quasi-category 
        \[ \mathscr P(\cat C):=\operatorname{Fun}(\cat C\op,\cat S) \]
        is called the category of \term{presheaves} on $\cat C$,
        where $\cat S:=\mathfrak N(\cat{Kan})$.
    
        \item The map
        \[\begin{aligned}
            Y\:\cat C&\to\mathscr P(\cat C),\\
            x&\mapsto h_x:=\Hom_{\cat C}(-,x)
        \end{aligned}\]
        is called the \term{Yoneda embedding}.
        Precisely, this map is defined by the simplicially enriched functor 
        \[\begin{aligned}
            \mathfrak C(\cat C\times\cat C\op)&\to\cat{Kan},\\
            (x,y)&\mapsto R\Hom_{\mathfrak C\cat C}(y,x)
        \end{aligned}\]
        by passing to the adjoint,
        where $R$ denotes the fibrant replacement.
    
        \item A presheaf on $\cat C$ is called \term{representable} if 
        it is equivalent to $h_x$ for some $x\in\cat C$.
    \end{itms}
\end{definition}

Via the Grothendieck construction,
presheaves on $\cat C$ are equivalent to right fibrations over $\cat C$.
Namely, given $F\in\mathscr P(\cat C)$, the right fibration 
\[ \operatorname{Un}_{\cat C}F\to\cat C \]
has fibre (equivalent to) $F(x)$ at $x\in\cat C$.

\begin{definition}
    Let $\cat C$ be a quasi-category.
    A right fibration $X\to\cat C$ is \term{representable}, 
    if the presheaf 
    \[ \operatorname{St}_{\cat C}X\in\mathscr P(\cat C) \]
    is representable.
\end{definition}

By definition, right fibrations of the form $\operatorname{Un}_{\cat C}h_x$
are representable.

Dually, a left fibration $p\:X\to\cat C$ is representable 
if the right fibration $p\op\:X\op\to\cat C\op$ is representable.

\begin{proposition}
    Let $\cat C$ be a quasi-category,
    and let $p\:\tilde{\cat C}\to\cat C$ be a right fibration.
    \begin{itms}
        \item $p$ is representable, if and only if $\tilde{\cat C}$
        has a terminal object.
        \item An object $x\in\cat C$ represents $p$,
        if and only if there exists a terminal object $\tilde x$ of $\tilde{\cat C}$,
        with $p(\tilde x)=x$.
    \end{itms}
    In particular, the full subcategory of $\cat C$
    spanned by the representing objects of $p$
    is either empty or a contractible Kan complex.
\end{proposition}

\begin{proof}
    \nyw
\end{proof}

We can now give the equivalent definitions
of adjoint functors.

\begin{proposition}
    Let $\cat C,\cat D$ be two quasi-categories.
    Then a pair of adjoint functors between $\cat C$ and $\cat D$
    is equivalent to a left fibration 
    \[ p\:X\to\cat C\op\times\cat D, \]
    such that 
    \begin{itms}
        \item $p$ is \term{left representable}:
        for any $x\in\cat C$, the slice 
        \[ p|_{x\times\cat D}\:p^{-1}(x\times\cat D)\to x\times\cat D \]
        is a representable left fibration.
        \item $p$ is \term{right representable}:
        for any $y\in\cat D$, the slice 
        \[ p|_{\cat C\op\times y}\:p^{-1}(\cat C\op\times y)\to \cat C\op\times y \]
        is a representable left fibration.
    \end{itms}
    In fact, $p$ is given by the Grothendieck construction 
    of the functor
    \[ \Hom_{\cat D}(F-,-)\simeq\Hom_{\cat C}(-,G-)\:\cat C\op\times\cat D\to\cat S. \]
\end{proposition}

\begin{proof}[Sketch of Proof]
    \nyw
\end{proof}

\begin{proposition}
    Let $\cat C,\cat D$ be two quasi-categories.
    Then a pair of adjoint functors between $\cat C$ and $\cat D$
    is equivalent to a map 
    \[ p\:M\to\Delta[1], \]
    such that 
    \begin{itms}
        \item There are DK equivalences
        $p^{-1}(0)\simeq\cat C$ and $p^{-1}(1)\simeq\cat D$.
        \item $p$ is a cocartesian fibration.
        \item $p$ is a cartesian fibration.
    \end{itms}
    In fact, $M$ is given by $\mathfrak N\cat M$,
    where the $\cat{Kan}$-enriched category $\cat M$ is defined by 
    \begin{itms}
        \item Objects: pairs $(x,0)$ for $x\in\cat C$ and $(y,1)$ for $y\in\cat D$.
        \item Morphism spaces:
        \[ \begin{aligned}
            \Hom_{\cat M}((x,0),(y,0)) &:= \Hom_{R\mathfrak C\cat C}(x,y), \\
            \Hom_{\cat M}((x,1),(y,1)) &:= \Hom_{R\mathfrak C\cat D}(x,y), \\
            \Hom_{\cat M}((x,0),(y,1)) &:= \Hom_{R\mathfrak C\cat D}(Fx,y)\simeq\Hom_{R\mathfrak C\cat C}(x,Gy), \\
            \Hom_{\cat M}((x,1),(y,0)) &:= \emptyset.
        \end{aligned} \]
    \end{itms}
\end{proposition}

\begin{proof}[Sketch of Proof]
    \nyw
\end{proof}

\begin{proposition}
    Let $\cat K,\cat C$ be quasi-categories.
    Suppose that every functor $\cat K\to\cat C$ has a colimit.
    Then the constant functor 
    \[ \operatorname{const}\:\cat C\to\operatorname{Fun}(\cat K,\cat C) \]
    admits a left adjoint, which is the colimit functor.
\end{proposition}

\begin{proof}
    \nyw
\end{proof}

\begin{remark}
    Let $\cat C,\cat D$ be simplicial model categories,
    and let $(F\dashv G)$ be a Quillen adjunction between them.
    Then $F$ and $G$ give rise to 
    a pair of adjoint functors between the quasi-categories
    $\mathfrak N(\cat C_{\mathrm{cf}})$ and
    $\mathfrak N(\cat D_{\mathrm{cf}})$. \varqed
\end{remark}
