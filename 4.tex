Infinity categories are difficult to study.
Unlike in classical category theory,
in order to understand infinity categories,
one needs to work with various models, instead of a single definition.

In these notes, we will work with two models for $(\infty,1)$-categories:
quasi-categories, and categories enriched over Kan complexes.
We have already defined both concepts in previous sections,
and we have seen why they are able to model infinity categories.
Now, we will explore the relationships between these two models.
We will see that these models are equivalent,
in the sense of a Quillen equivalence.

\subsection{Quasi-categories}

Recall from the previous section that a quasi-category is a simplicial set in which
inner horns can be extended.

Let $\cat C$ be a quasi-category, 
Let $x,y\in\cat C_0$ be two points.
Our first goal is to define the hom-space $\Hom_{\cat C}(x,y)$.
Instead of being a discrete set, it should contain information
describing homotopies and higher homotopies, 
which comprise the higher structure of an infinity category.

We now introduce some terminology for
a quasi-category $\cat C$.
\begin{itms}
    \item We say $x\in\cat C$ is an \term{object} of $\cat C$, if $x\in\cat C_0$.
    \item We say $f\:x\to y$ is a \term{morphism} of $\cat C$, if $f\in\cat C_1$,
    and $d_0f=x$, $d_1f=y$.
    \item The \term{identity morphism} $1_x$ of an object $x\in\cat C$
    refers to the degenerate $1$-simplex $s_0(x)$.
    \item For morphisms $f\:x\to y$ and $g\:y\to z$,
    we say that a morphism $h\:x\to z$
    is a \term{composition} of $g$ and $f$, if 
    there is a $2$-simplex $\sigma\in\cat C_2$ such that 
    $d_0\sigma=g$, $d_2\sigma=f$, and $d_1\sigma=h$.
    This can be drawn as a diagram
    \[\begin{tikzcd}[row sep=3ex,column sep=.7em]
        & y\ar[dr,"g"] \\
        x\ar[ur,"f"]\ar[rr,"h"'] && z
    \end{tikzcd}\]
    % \[\xymatrix @R=1.5em @C=.7em{ & y\ar[dr]^g \\ x\ar[ur]^f\ar[rr]_h && z }\]
    in $\cat C$. Note that composition is not unique in a quasi-category.
\end{itms}

\begin{definition}
    Let $f,g\:x\to y$ be two morphisms in $\cat C$.
    We say that $f$ and $g$ are \term{homotopic},
    if the following equivalent conditions hold.
    \begin{itms}
        \item There is a $2$-simplex \enspace
        \begin{tikzcd}[row sep=.5ex,column sep=1.5em]
            x\ar[dd,"1_x"']\ar[dr,pos=.3,"f"] \\
            & y \\
            x\ar[ur,pos=.3,"g"']
        \end{tikzcd}
        % \vcenter{\xymatrix @R=.2em @C=1.5em { x\ar[dd]_{1_x}\ar[dr]^(.4)f \\ & y \\ x\ar[ur]_(.4)g }}
        \enspace in $\cat C$.
        \item There is a $2$-simplex \enspace
        \begin{tikzcd}[row sep=.5ex,column sep=1.5em]
            x\ar[dd,"1_x"']\ar[dr,pos=.3,"g"] \\
            & y \\
            x\ar[ur,pos=.3,"f"']
        \end{tikzcd}
        % \vcenter{\xymatrix @R=.2em @C=1.5em { x\ar[dd]_{1_x}\ar[dr]^(.4)g \\ & y \\ x\ar[ur]_(.4){\vphantom{g}\smash{f}} }}
        \enspace in $\cat C$.
        \item There is a $2$-simplex \enspace
        \begin{tikzcd}[row sep=.5ex,column sep=1.5em]
            & y\ar[dd,"1_y"] \\
            x\ar[ur,pos=.7,"f"]\ar[dr,pos=.7,"g"'] \\
            & y
        \end{tikzcd}
        % \vcenter{\xymatrix @R=.2em @C=1.5em { & y\ar[dd]^{1_y} \\ x\ar[ur]^(.6)f\ar[dr]_(.6)g \\ & y }}
        \enspace in $\cat C$.
        \item There is a $2$-simplex \enspace
        \begin{tikzcd}[row sep=.5ex,column sep=1.5em]
            & y\ar[dd,"1_y"] \\
            x\ar[ur,pos=.7,"g"]\ar[dr,pos=.7,"f"'] \\
            & y
        \end{tikzcd}
        % \vcenter{\xymatrix @R=.2em @C=1.5em { & y\ar[dd]^{1_y} \\ x\ar[ur]^(.6)g\ar[dr]_(.6){\vphantom{g}\smash{f}} \\ & y }}
        \enspace in $\cat C$.
        \item There is a square \enspace
        \begin{tikzcd}[row sep=2.5ex,column sep=1.5em]
            x\ar[d,"1_x"']\ar[r,"f"] & y\ar[d,"1_y"] \\
            x\ar[r,"g"'] & y
        \end{tikzcd}
        % \vcenter{\xymatrix @R=1.2em @C=1.5em { x\ar[d]_{1_x}\ar[r]^f & y\ar[d]^{1_y} \\ x\ar[r]_g & y }}
        \enspace in $\cat C$, which is a map $\Delta[1]\times\Delta[1]\to\cat C$.
        \item There is a square \enspace
        \begin{tikzcd}[row sep=2.5ex,column sep=1.5em]
            x\ar[d,"1_x"']\ar[r,"g"] & y\ar[d,"1_y"] \\
            x\ar[r,"f"'] & y
        \end{tikzcd}
        % \vcenter{\xymatrix @R=1.2em @C=1.5em { x\ar[d]_{1_x}\ar[r]^g & y\ar[d]^{1_y} \\ x\ar[r]_{\vphantom{g}\smash{f}} & y }}
        \enspace in $\cat C$.
    \end{itms}
\end{definition}

Using the horn extension property for $\Lambda_1[3]$ and $\Lambda_2[3]$,
one can show that all the above conditions are equivalent.
We leave this as an exercise for the reader.

\begin{proposition}
    Homotopy of morphisms is an equivalence relation,
    and respects composition of morphisms.
    In particular, composition is unique up to homotopy.
\end{proposition}

\begin{proof}
    Exercise for the reader.
\end{proof}

\begin{definition}
    Let $\cat C$ be a quasi-category.
    The \term{homotopy category} of $\cat C$
    is an ordinary category $\Ho(\cat C)$, defined as follows.
    \begin{itms}
        \item Its objects are objects of $\cat C$.
        \item Its morphisms are homotopy classes of morphisms in $\cat C$.
    \end{itms}
\end{definition}

We regard $\Ho(\cat C)$ as obtained from $\cat C$
by forgetting its higher structure.

\begin{theorem}[Joyal]\label{thm-4-j}
    Let $\cat C$ be a quasi-category.
    Then $\cat C$ is a Kan complex if and only if $\Ho(\cat C)$ is a groupoid.
\end{theorem}

It is easy to see that if $\cat C$ is a Kan complex, i.e.\ an $\infty$-groupoid,
then $\Ho(\cat C)$ is a groupoid.
But the converse is a non-trivial result.

Our next step is to assign a ``higher structure'', i.e.\ a homotopy type,
to every hom-space $\Hom_{\cat C}(x,y)$,
as a way to describe higher homotopies between morphisms.

\begin{definition}
    Let $\cat C$ be a quasi-category, and let $x,y\in\cat C$ be two objects.

    \begin{itms}
        \item The simplicial set $\Hom^{\vartriangleright}_{\cat C}(x,y)$ is defined as follows.
        Its $n$-simplices are 
        \[\Hom^{\vartriangleright}_{\cat C}(x,y)_n:=
        \bigl\{\sigma\in\cat C_{n+1}\bigmid\sigma|_{\Delta\{0,\dotsc,n\}}=x,\ \sigma(n+1)=y\bigr\},\]
        where $\sigma$ is regarded as a map $\Delta[n+1]\to\cat C$,
        and $\Delta\{0,\dotsc,n\}\subset\Delta[n+1]$ is the face spanned by the vertices $0,\dotsc,n$.
    
        \item Dually, we define the simplicial set $\Hom^{\vartriangleleft}_{\cat C}(x,y)$ by
        \[\Hom^{\vartriangleleft}_{\cat C}(x,y)_n:=
        \bigl\{\sigma\in\cat C_{n+1}\bigmid\sigma(0)=x,\ \sigma|_{\Delta\{1,\dotsc,n+1\}}=y\bigr\}.\]
    
        \item There is yet another version of the hom-space
        $\Hom_{\cat C}^{\square}(x,y)$, defined by
        \[\Hom^{\square}_{\cat C}(x,y)_n:=
        \bigl\{\sigma\:\Delta[n]\times\Delta[1]\to\cat C\bigmid
        \sigma|_{\Delta[n]\times\{0\}}=x,\ \sigma|_{\Delta[n]\times\{1\}}=y\bigr\}.\]
    \end{itms}
\end{definition}

These three constructions do not give isomorphic simplicial sets in general.
However, we will see that these three simplicial sets 
are homotopy equivalent Kan complexes,
so that the hom-space has a well-defined homotopy type.

Note that none of these constructions can produce a category 
enriched over simplicial sets.
The reason is that composition can not be well-defined.
However, we will soon construct a simplicial category 
whose hom-spaces are homotopy equivalent to these ones.

\begin{proposition}
    The simplicial sets $\Hom^{\vartriangleright}_{\cat C}(x,y)$,
    $\Hom^{\vartriangleleft}_{\cat C}(x,y)$ and
    $\Hom^{\square}_{\cat C}(x,y)$ are Kan complexes.
\end{proposition}

\begin{proof}
    For $\Hom^{\vartriangleright}_{\cat C}(x,y)$,
    it is not difficult to see that it has the horn extension property 
    for the horns $\Lambda_i[n]$ for $0<i\leq n$.
    Therefore, it is a quasi-category.
    Moreover, in its homotopy category, every morphism has a left inverse.
    This implies that its homotopy category is a groupoid,
    so that by (\ref{thm-4-j}), it is a Kan complex.
    The same argument shows that
    $\Hom^{\vartriangleleft}_{\cat C}(x,y)$ is also a Kan complex.

    For $\Hom^{\square}_{\cat C}(x,y)$,
    the proof uses the Joyal model structure on $\cat{sSet}$,
    and will be presented later.
\end{proof}

\subsection{Simplicial categories}

Recall from the first section that intuitively,
an $(\infty,1)$-category can be seen as a category enriched over
the category of $(\infty,0)$-categories,
which are modelled with Kan complexes.
Therefore, categories enriched over Kan complexes
should be another model for $(\infty,1)$-categories.
We generalise this a bit by considering categories 
enriched over all simplicial sets.

\begin{definition}
    A \term{simplicial category} is a category 
    enriched over $\cat{sSet}$.
\end{definition}

In a simplicial category,
we regard the $0$-simplices of the hom-spaces as morphisms,
the $1$-simplices as homotopies between morphisms,
and higher dimensional simplices as higher homotopies.

For example, the category $\cat{sSet}$ is a simplicial category,
equipped with the following simplicial structure on its hom-spaces.

\begin{definition}
    Let $X,Y$ be two simplicial sets.
    The \term{mapping space} $\operatorname{Map}(X,Y)$
    is the simplicial set whose $n$-simplices are maps
    from $X\times\Delta[n]$ to $Y$.
\end{definition}

The category $\cat{Top}$ can also be seen as a simplicial category,
by taking the $\operatorname{Sing}$ of all its mapping spaces,
equipped with the compact open topology.

It is very easy to define the homotopy category of a simplicial category.

\begin{construction}
    Let $\cat C$ be a simplicial category. The functors
    \[ \cat{sSet}\xrightarrow{h}\cat{hCW}\xrightarrow{\pi_0}\cat{Set} \]
    assign $\cat C$ with an $\cat{hCW}$-enriched category $h\cat C$,
    and an ordinary category denoted by $\Ho(\cat C):=\pi_0h\cat C$.
    The latter is called the \term{homotopy category} of $\cat C$. \varqed
\end{construction}

Although the definition of an enriched category requires
strict (i.e.\ unique) composition and strict associativity,
many other things can be done in the non-strict, or ``up to homotopy'' way,
and it is often good to think of simplicial categories in the non-strict way.
Here is an example.

\begin{definition}
    A \term{simplicial groupoid} is a simplicial category
    whose homotopy category is a groupoid.
    A \term{simplicial group} is a simplicial groupoid with a unique object.
\end{definition}

Compare (\ref{thm-4-j}).
In this definition, although composition is strict,
taking the inverse is non-strict, i.e.\ up to homotopy.

Next, we aim to define a pair of adjoint functors
\[\begin{tikzcd}
    \cat{sSet}\ar[r,bend left,"\mathfrak C"]\ar[r,phantom,"\bot"] &
    \rlap{$\cat{Cat}_{\cat{sSet}}$\ ,}\phantom{\cat{sSet}}\ar[l,bend left,"\mathfrak N"]
\end{tikzcd}\]
% \[\xymatrix { \cat{sSet}\ar@/^2ex/[r]^{\mathfrak C}\ar@{}[r]|{\bot} & \rlap{\cat{Cat}_{\cat{sSet}}\ ,}\phantom{\cat{sSet}}\ar@/^2ex/[l]^{\mathfrak N} }\hspace{1.2em}\]
as a conversion between quasi-categories and simplicial categories.
We will see that this adjunction is a Quillen equivalence,
given suitable model structures on both sides.

To define such an adjunction, by (\ref{con-3-a}),
we only need to specify what $\mathfrak C\Delta[n]$ is.
Intuitively, it should look like a chain of $n$ arrows
\[0\to1\to\cdots\to n\ ,\]
whose nerve is $\Delta[n]$.
However, we need to modify it in order to allow 
the composition law, e.g.\ $(0\to1\to2)=(0\to2)$,
to hold only up to homotopy. This motivates the following construction.

\begin{construction}
    Let $n\geq0$ be an integer. The simplicial category $\mathfrak C^n$ is defined as follows.
    \begin{itms}
        \item It has $(n+1)$ objects, which we call $0,1,\dotsc,n$.
        \item Its hom-spaces are given by $\Hom(i,j)=N(P_{ij})$, where 
        \[P_{ij}:=\begin{cases}
            \ \emptyset, & i>j,\\
            \ \text{poset of subsets of $\{i,i+1,\dotsc,j\}$ containing $i,j$}, & i\leq j,
        \end{cases}\]
        where a poset is naturally regarded as a category.
        Composition is defined by union of sets. \varqed
    \end{itms}
\end{construction}

For example, $\mathfrak C^2$ is the simplicial category 
\[\begin{tikzcd}[column sep=2em]
    & 1 \arrow{dr}[inner sep=0pt]{
    \begin{tikzpicture}[every node/.style={inner sep=1pt},
                        every label/.style={inner sep=0pt}]
        \node[label=right:12]{\bullet};
    \end{tikzpicture}} \\
    0 \arrow{ur}[inner sep=0pt]{
    \begin{tikzpicture}[every node/.style={inner sep=1pt},
                        every label/.style={inner sep=0pt}]
        \node[label=left:01]{\bullet};
    \end{tikzpicture}}
    \arrow{rr}[swap]{
    \begin{tikzpicture}[node distance=2.5em,
                        every node/.style={inner sep=1pt},
                        every label/.style={inner sep=0pt}]
        \node(02)[label=below:02]{\bullet}; 
        \node(012)[right of=02,label=below:012]{\bullet};
        \draw[->](02)--(012);
    \end{tikzpicture}}
    && 2\rlap{\ ,}
\end{tikzcd}\]
where the composition $12\circ01$ gives the map $012\:0\to2$,
which is homotopic, but not equal to, the map $02\:0\to2$
which we regard as going directly (not passing $1$) from $0$ to $2$.

Likewise, $\mathfrak C^3$ is the simplicial category
\[\begin{tikzcd}[row sep=6em,column sep=4em]
    & 1 \arrow{rr}[inner sep=0pt]{
    \begin{tikzpicture}[every node/.style={inner sep=1pt},
                        every label/.style={inner sep=0pt}]
        \node[label=above:12]{\bullet};
    \end{tikzpicture}}
    \arrow[start anchor=south east]{drrr}[pos=.5,swap,inner sep=0pt]{
    \begin{tikzpicture}[node distance=2em,
                        every node/.style={inner sep=1pt},
                        every label/.style={inner sep=0pt}]
        \node(13)[label=left:13]{\bullet}; 
        \node(123)[below of=13,label=left:123]{\bullet};
        \draw[->](13)--(123);
    \end{tikzpicture}}
    && 2 \arrow{dr}[inner sep=0pt]{
    \begin{tikzpicture}[every node/.style={inner sep=1pt},
                        every label/.style={inner sep=0pt}]
        \node[label=right:23]{\bullet};
    \end{tikzpicture}} \\
    0 \arrow{ur}[inner sep=0pt]{
    \begin{tikzpicture}[every node/.style={inner sep=1pt},
                        every label/.style={inner sep=0pt}]
        \node[label=left:01]{\bullet};
    \end{tikzpicture}}
    \arrow[crossing over,crossing over clearance=1ex,end anchor=south west]{urrr}
    [pos=.4,swap,inner sep=0pt]{
    \begin{tikzpicture}[node distance=2em,
                        every node/.style={inner sep=1pt},
                        every label/.style={inner sep=0pt}]
        \node(02)[label=below:02]{\bullet}; 
        \node(012)[right of=02,label=below:012]{\bullet};
        \draw[->](02)--(012);
    \end{tikzpicture}}
    \arrow{rrrr}[swap]{
    \begin{tikzpicture}[node distance=2em,
                        every node/.style={inner sep=1pt},
                        every label/.style={inner sep=0pt}]
        \node(03)[label=left:03]{\bullet}; 
        \node(023)[below of=03,label=left:023]{\bullet};
        \node(013)[right of=03,label=right:013]{\bullet}; 
        \node(0123)[below of=013,label=right:0123]{\bullet};
        \draw[->](03)--(013);
        \draw[->](03)--(023);
        \draw[->](013)--(0123);
        \draw[->](023)--(0123);
        \draw[->](03)--(0123);
    \end{tikzpicture}}
    &&&& 3\rlap{\ ,}
\end{tikzcd}\]
where the mapping space $N(P_{03})$
is isomorphic to $\Delta[1]\times\Delta[1]$.
As we can see, the $4$ points in this space 
correspond to the $4$ ways to go from $0$ to $4$ along the arrows,
namely, $0\to3$, $0\to1\to3$, $0\to2\to3$, and $0\to1\to2\to3$.
All these $4$ maps from $0$ to $3$ are homotopic.

In general, the mapping space $\Hom_{\mathfrak C^n}(i,j)$
is the space of all ways to go from $i$ to $j$ along the arrows.
If $i<j$, then it is isomorphic to $\Delta{[1]}^{j-i-1}$. 

\begin{definition}
    The cosimplicial object $\mathfrak C^\bullet$ in $\cat{Cat}_{\cat{sSet}}$
    determines an adjunction
    \[\begin{tikzcd}
        \cat{sSet}\ar[r,bend left,"\mathfrak C"]\ar[r,phantom,"\bot"] &
        \rlap{$\cat{Cat}_{\cat{sSet}}$\ ,}\phantom{\cat{sSet}}\ar[l,bend left,"\mathfrak N"]
    \end{tikzcd}\]
    % \[\xymatrix { \cat{sSet}\ar@/^2ex/[r]^{\mathfrak C}\ar@{}[r]|{\bot} & \rlap{\cat{Cat}_{\cat{sSet}}\ ,}\phantom{\cat{sSet}}\ar@/^2ex/[l]^{\mathfrak N} }\hspace{1.2em}\]
    in the sense of \textup{(\ref{con-3-a})}.
    The right adjoint $\mathfrak N$ is called the \term{simplicial nerve} functor,
    or the \term{homotopy coherent nerve} functor.
\end{definition}

Now we can prove that categories enriched over Kan complexes 
are converted to quasi-categories under this construction.

\begin{proposition}
    If $\cat C$ is a category enriched over Kan complexes,
    then $\mathfrak N\cat C$ is a quasi-category.
\end{proposition}

\begin{proof}
    \nyw
\end{proof}

\subsection{Equivalence of the two models}

For a quasi-category or a simplicial category $\cat C$,
we have defined a mapping space $\Hom_{\cat C}(x,y)$ for each
pair of objects $x,y$ in $\cat C$.
In fact, the homotopy type of the mapping space 
is preserved by the above conversion between the two models.

\begin{theorem}\label{thm-4-h}
    We have the following.
    \begin{itms}
        \item If $\cat C$ is a quasi-category,
        then we have a weak equivalence of simplicial sets 
        \[ \Hom_{\cat C}^{(*)}(x,y)\simeq\Hom_{\mathfrak C\cat C}(x,y) \]
        for each pair of objects $x,y\in\cat C$,
        where $(*)$ can be $\vartriangleleft$, $\vartriangleright$, or $\square$.
        \item If $\cat C$ is a category enriched over Kan complexes,
        then we have a homotopy equivalence of Kan complexes
        \[ \Hom_{\cat C}(x,y)\simeq\Hom_{\mathfrak N\cat C}^{(*)}(x,y) \]
        for each pair of objects $x,y\in\cat C$,
        where $(*)$ can be $\vartriangleleft$, $\vartriangleright$, or $\square$.
    \end{itms}
\end{theorem}

The proof of the theorem will be given in the next section.
Note that weak equivalences between Kan complexes
are homotopy equivalences,
by Whitehead's theorem (\ref{thm-2-w}).

Thus, for a quasi-category $\cat C$, we can define the 
$\cat{hCW}$-enriched category $h\cat C$ to be $h\mathfrak C\cat C$.
It has the same objects and the same homotopy types of mapping spaces as $\cat C$.

Next, we describe the adjoint pair $(\mathfrak C\dashv\mathfrak N)$
as a Quillen equivalence,
so that the two models are really equivalent.

\begin{definition}
    Let $\cat C,\cat D$ be two quasi-categories, or two simplicial categories.
    A map $f\:\cat C\to\cat D$ is called a \term{Dwyer--Kan equivalence},
    or a \term{categorical equivalence}, if $h(f)\:h\cat C\to h\cat D$
    is an equivalence of $\cat{hCW}$-enriched categories, i.e.\ the following holds.
    \begin{itms}
        \item $f$ is fully faithful, i.e.\ induces weak equivalences of mapping spaces.
        \item $f$ is essentially surjective, i.e.\ $\pi_0h(f)=\Ho(f)$
        is an essentially surjective functor between ordinary categories.
    \end{itms}
\end{definition}

\begin{remark}
    Dwyer--Kan equivalences are the $\infty$-categorical notion
    of a categorical equivalence.
    We regard two $\infty$-categories as equivalent,
    whenever they are Dwyer--Kan equivalent. \varqed
\end{remark}

Recall that a functor $f\:\cat C\to\cat D$ between ordinary categories is an \term{isofibration},
if for any $x\in\cat C$ and any isomorphism $\alpha\:f(x)\to y$ in $\cat D$,
there exists $\tilde\alpha\:x\to\tilde y$ in $\cat C$ such that $\alpha=f(\tilde\alpha)$.

\begin{theorem}\label{thm-4-e}
    The category $\cat{sSet}$ has the \term{Joyal model structure}, with 
    \begin{itms}
        \item $\cat W=\{$Dwyer--Kan equivalences$\}$.
        \item $\cat{Cof}=\{$injections$\}$.
        \item The fibrant objects are quasi-categories.
    \end{itms}
    The category $\cat{Cat}_{\cat{sSet}}$ has the \term{Bergner model structure}, with
    \begin{itms}
        \item $\cat W=\{$Dwyer--Kan equivalences$\}$.
        \item $\cat{Fib}=\{$isofibrations that are fibrations on mapping spaces$\}$.
        \item The fibrant objects are categories enriched over Kan complexes.
    \end{itms}
    Using these model structures, the adjunction
    \[\begin{tikzcd}
        \cat{sSet}\ar[r,bend left,"\mathfrak C"]\ar[r,phantom,"\bot"] &
        \rlap{$\cat{Cat}_{\cat{sSet}}$}\phantom{\cat{sSet}}\ar[l,bend left,"\mathfrak N"]
    \end{tikzcd}\]
    is a Quillen equivalence.
\end{theorem}

For a proof, see \cite[Theorem~2.2.5.1]{htt}.

Since $\mathfrak N$ is right Quillen,
it sends category enriched over Kan complexes to quasi-categories.
However, $\mathfrak C$ does not necessarily
send a quasi-category to a category enriched over Kan complexes.
Given a quasi-category $\cat C$,
its corresponding category enriched over Kan complexes should be 
\[ R\mathfrak C\cat C, \]
where $R$ denotes the fibrant replacement functor of $\cat{sSet}$.
For example, one may take $R$ to be the functor $\operatorname{Sing}|\cdot|$.

Let $\cat C$ be a quasi-category, and let $\cat D$ be a category 
enriched over Kan complexes. The unit and counit maps
\[ \mathfrak NR\mathfrak C\cat C\to\cat C\quad\text{and}\quad\cat D\to\mathfrak CQ\mathfrak N\cat D \]
are always Dwyer--Kan equivalences,
as is a standard fact in model category theory \cite[Theorem~1.3.13]{hovey}.
Here we may take $Q$ to be the identity functor, 
which implies that the functors
\[ \mathfrak NR\mathfrak C\cat C\to\cat C\quad\text{and}\quad\cat D\to R\mathfrak C\mathfrak N\cat D \]
are Dwyer--Kan equivalences.

This discussion shows that one can convert between
quasi-categories and categories enriched over Kan complexes,
using the functors $R\mathfrak C$ and $\mathfrak N$,
in a way that preserves Dwyer--Kan equivalences.

\begin{remark}
    We do not use the standard model structure on $\cat{sSet}$
    when we study quasi-categories, and here is a reason.
    Let $\cat C$ be a quasi-category.
    In many cases, $\cat C$ contains an initial object $x$,
    and thus, the geometric realisation of $\cat C$ 
    can be contracted to $x$ linearly.
    Therefore, the homotopy type of $\cat C$ is trivial.
    On the other hand, the Joyal model structure 
    captures the internal information of a quasi-category,
    rather than the global homotopy type. \varqed
\end{remark}

\subsection{Examples}

Let us look at a few examples of $(\infty,1)$-categories,
modelled as quasi-categories.
We are now able to talk about these objects in a rigorous way.

\begin{proposition}\label{thm-4-x}
    Let $X,Y$ be two simplicial sets.
    \begin{itms}
        \item If $Y$ is a Kan complex,
        then $\operatorname{Map}(X,Y)$ is a Kan complex.
        \item If $Y$ is a quasi-category,
        then $\operatorname{Map}(X,Y)$ is a quasi-category.
    \end{itms}
\end{proposition}

\begin{proof}
    Exercise for the reader.
\end{proof}

\begin{example}
    The \term{quasi-category of spaces} is defined to be
    \[\cat S:=\mathfrak N(\cat{Kan}),\]
    where $\cat{Kan}$ is the simplicial category of Kan complexes,
    which is enriched over Kan complexes by (\ref{thm-4-x}).
    \varqed
\end{example}

\begin{example}
    The $\cat{Kan}$-enriched category $\cat{QsCat}$
    is defined as follows.
    \begin{itms}
        \item The objects are quasi-categories.
        \item The morphism space $\Hom(\cat C,\cat D)$
        is the maximal Kan complex in the simplicial set $\operatorname{Map}(\cat C,\cat D)$.
    \end{itms}
    Such a maximal Kan complex exists,
    since by (\ref{thm-4-x}) and (\ref{thm-4-j}),
    it is the sub-simplicial set spanned by all the invertible edges.

    The \term{quasi-category of quasi-categories} is defined to be 
    \[\cat{Cat}_\infty:=\mathfrak N(\cat{QsCat}).\]

    Note that we have taken the maximal Kan complex, instead of a fibrant replacement.
    This is because the edges in the mapping space are natural transformations,
    and we discard those natural transformations that are not invertible,
    rather than inverting them.

    This means that in order to get an $(\infty,1)$-category of $(\infty,1)$-categories,
    we have to discard some information.
    The essential reason is that $(\infty,1)$-categories should form an $(\infty,2)$-category.
    This can be seen as follows:
    in our model, the simplicial category of quasi-categories
    is naturally enriched over quasi-categories,
    making it an $(\infty,2)$-category. \varqed
\end{example}

\begin{example}
    Let $\cat A$ be an abelian category.
    The simplicial category $\cat{Ch}_{\cat A}$ of cochain complexes in $\cat A$
    is defined as follows.
    Recall that for a map $f\:X\to Y$ of cochain complexes (not necessarily a chain map),
    we defined 
    \[ df:=d\circ f-(-1)^{|f|}f\circ d, \]
    where $|f|=k$ if $f$ sends $X^n$ to $Y^{n+k}$.
    \begin{itms}
        \item The $0$-simplices of $\Hom(X,Y)$ are those maps $f\:X\to Y$
        such that 
        \[ |f|=0\quad\text{and}\quad df=0. \]
        In other words, they are chain maps.
        \item A $1$-simplex between two $0$-simplices $f,g$ is a map $a\:X\to Y$
        such that 
        \[ |a|=-1\quad\text{and}\quad da=f-g. \]
        In other words, they are chain homotopies.
        \item An $n$-simplex consists of the data $(\sigma,\sigma_0,\dotsc,\sigma_n)$,
        where each $\sigma_i$ is an $(n-1)$-simplex,
        and $\sigma\:X\to Y$ is a map satisfying
        \[ |\sigma|=-n\quad\text{and}\quad d\sigma=\sigma_0-\sigma_1+\sigma_2-\cdots\pm\sigma_n. \]
        The $\sigma_i$ are the faces of $\sigma$, and we require that
        the faces of the $\sigma_i$ are compatible with each other.
        In other words, an $n$-simplex is a map 
        \[ C_\bullet^{\textrm{cell}}(\Delta^n)\to\sHom(X,Y) \]
        from the cellular chain complex of $\Delta^n$
        to the chain complex $\sHom(X,Y)$ defined in (\ref{eg-1-c}).
    \end{itms}
    We define 
    \[\cat K_\infty(\cat A):=\mathfrak N(\cat{Ch}_{\cat A})\]
    to be the \term{quasi-category of cochain complexes} in $\cat A$.
    If $\cat{Ch}_{\cat A}$ has a suitable model structure,
    e.g.\ if $\cat A$ is the category of modules over a ring, then we define 
    \[\cat D_\infty(\cat A):=\mathfrak N(\cat{Ch}_{\cat A,\mathrm{cf}})\]
    to be the \term{derived quasi-category} of $\cat A$. \varqed
\end{example}

\begin{remark}
    We have seen that localising a category with weak equivalences
    gives rise to $\infty$-categories.
    It turns out that if $\cat C$ is a model category,
    together with a simplicial enrichment,
    satisfying some extra compatibility conditions,
    then we have an equivalence of quasi-categories
    \[ \cat C[\cat W^{-1}]\simeq\mathfrak N(\cat C_{\mathrm{cf}}). \]
    Such a simplicial structure can be found in all the examples that we have seen.
    Therefore, we have
    \[ \cat S\simeq\cat{sSet}[\cat W^{-1}]\quad\text{and}\quad
    \cat D_\infty(\cat A)\simeq\cat{Ch}_{\cat A}[\cat W^{-1}], \]
    as examples of $\infty$-categorical localisations. \varqed
\end{remark}
