In this section, our main goal is to study the category of cochain complexes.
We will construct the $\infty$-category of cochain complexes, and then, 
we will invert the quasi-isomorphisms in it, obtaining an $\infty$-categorical
version of the derived category.

\subsection{Nerve of a dg category}

\begin{definition}
    Let $R$ be a commutative ring.
    A \term{dg category} over $R$ is a category enriched over the
    symmetric monoidal category $(\cat{Ch}_R,\otimes)$
\end{definition}

The abbreviation ``dg'' stands for ``differential graded''.
Cochain complexes are also known as dg vector spaces (or modules).

\begin{definition}
    Let $\cat{C}$ be a dg category over $R$.
    \begin{itms}
        \item The \term{underlying category} of $\cat{C}$
        is obtained from $\cat{C}$ by applying the functor 
        \[Z^0\:(\cat{Ch}_R,\otimes)\to(\cat{Set},\times),\]
        taking the set of $0$-cocycles of a cochain complex.
        \item The \term{homotopy category} of $\cat{C}$
        is obtained from $\cat{C}$ by applying the functor 
        \[H^0\:(\cat{Ch}_R,\otimes)\to(\cat{Set},\times),\]
        taking the $0$-th cohomology of a cochain complex.
    \end{itms}
\end{definition}

For example, if $\cat{C}=\cat{Ch}_R$,
which is enriched over itself, as described in (\ref{eg-1-c}),
then the underlying category of $\cat{C}$ is just the ordinary
category of cochain complexes $\cat{Ch}_R$,
and its homotopy category is the ordinary category $\cat{hCh}_R$.
The $0$-coboundaries are exactly the chain homotopies.

We now wish to define a quasi-category $\Ndg(\cat{C})$,
such that the homotopies in this quasi-category are the chain homotopies,
and the higher homotopies are higher chain homotopies, etc.

A simple idea is to consider the sequence of adjunctions
\[\begin{tikzcd}
    \cat{sSet}\ar[r,bend left,start anchor=north east,end anchor=north west,"\mathfrak{C}"]
    \ar[r,phantom,"\bot"] &
    \cat{Cat}\smash{_{\cat{sSet}}}
    \ar[l,bend left,start anchor=south west,end anchor=south east,"\mathfrak{N}"]
    \ar[r,bend left,start anchor=north east,end anchor=north west,"\text{free}"]
    \ar[r,phantom,"\bot"] &
    \cat{Cat}\smash{_{\cat{sMod}_R}}
    \ar[l,bend left,start anchor=south west,end anchor=south east,"\text{forget}"]
    \ar[r,bend left,start anchor=north east,end anchor=north west,"\KD"]
    \ar[r,phantom,"\simeq"] &
    \cat{Cat}\smash{_{(\cat{Ch}_R)_{\geq0}}}
    \ar[l,bend left,start anchor=south west,end anchor=south east,"\operatorname{DK}"]
    \ar[r,bend left,start anchor=north east,end anchor=north west,"i_{\geq0}"]
    \ar[r,phantom,"\bot"] &
    \cat{Cat}\smash{_{\cat{Ch}_R}\rlap{\ ,}}
    \ar[l,bend left,start anchor=south west,end anchor=south east,"\tau_{\geq0}"]
\end{tikzcd}\]
and one may define 
\[ \Ndg:=\mathfrak{N}\circ\text{forget}\circ{\operatorname{DK}}\circ\tau_{\geq0}\:\cat{Cat}_{\cat{Ch}_R}\to\cat{sSet}, \]
and will see soon that the image lies in $\cat{QsCat}$.

This construction does give the correct quasi-category,
but there is a more direct way to construct this quasi-category.
We will first define $\Ndg$ in the more direct way,
and then we will prove that the two constructions are equivalent.

\begin{construction}
    Let $\cat{C}$ be a dg category.
    The quasi-category $\Ndg(\cat{C})$ is constructed as follows.
    Its $n$-simplices are of the form 
    \[(\{X_i\}_{i\in[n]},\{f_I\}_{I\subset[n],|I|\geq2}),\]
    where 
    \begin{itms}
        \item $X_0,\dotsc,X_n$ are objects of $\cat{C}$.
        \item For $0\leq i_0<\cdots<i_{m+1}\leq n$,
        \[f_{i_0\cdots i_{m+1}}\in\Hom_{\cat{C}}(X_{i_0},X_{i_{m+1}})_m,\]
        such that 
        \[df_{i_0\cdots i_{m+1}}=\sum_{j=1}^m(-1)^{m+1-j}
        \bigl(f_{i_0\cdots\widehat{i_j}\cdots i_{m+1}}-f_{i_j\cdots i_{m+1}}\circ f_{i_0\cdots i_j}\bigr).\]
    \end{itms}
    For example,
    \begin{itms}
        \item A $0$-simplex is just an object $X\in\cat{C}$.
        \item A $1$-simplex connecting the objects $X,Y\in\cat{C}$ is a morphism
        \[f_{01}\in\Hom_{\cat{C}}(X,Y)_0\quad\text{such that}\quad df_{01}=0.\]
        In other words, $f_{01}$ is just a morphism in the underlying category of $\cat{C}$.
        \item A $2$-simplex
        \[\begin{tikzcd}[column sep=1em]
            & Y\ar[dr,"g"] \\
            X\ar[ur,"f"]\ar[rr,"h"'] && Z
        \end{tikzcd}\]
        is a chain homotopy
        \[\alpha:=f_{012}\in\Hom_{\cat{C}}(X,Z)_1\quad\text{such that}\quad d\alpha=g\circ f-h.\]
    \end{itms}
    We leave it to the reader to define the face and degeneracy maps in $\Ndg(\cat{C})$,
    and verify that it is a quasi-category. \varqed
\end{construction}

\begin{proposition}
    Let $\cat{C}$ be a dg category. Then for any $X,Y\in\cat{C}$,
    there is an isomorphism of simplicial sets
    \[\Hom_{\Ndg(\cat{C})}^{\vartriangleright}(X,Y)\simeq\operatorname{DK}(\tau_{\geq0}\Hom_{\cat{C}}(X,Y)).\]
\end{proposition}

\begin{proof}
    \nyw
\end{proof}

\begin{corollary}
    Let $\cat{C}$ be a dg category. Then there is a categorical equivalence
    \[\Ndg(\cat{C})\simeq
    \mathfrak{N}\circ\mathrm{forget}\circ{\operatorname{DK}}\circ\tau_{\geq0}(\cat{C}).
    \qedhere\] \qed
\end{corollary}

\begin{theorem}
    The category $\cat{Cat}_{\cat{Ch}_R}$ has the \term{Tabuada model structure}, with
    \begin{itms}
        \item A weak equivalence is a functor inducing an equivalence of 
        homotopy categories, and inducing quasi-isomorphisms on all mapping spaces.
        \item A fibration is a functor inducing an isofibration of 
        homotopy categories, and inducing (degreewise) surjections on all mapping spaces.
    \end{itms}
    The functor
    \[\Ndg\:\cat{Cat}_{\cat{Ch}_R}\to\cat{sSet}\]
    is a right Quillen functor with respect to this model structure,
    and the Joyal model structure on $\cat{sSet}$.
\end{theorem}

\subsection{Derived categories}

\nyw 

\subsection{Derived categories as localisations}

\nyw 
