In topology, we have homotopy pushout and pullback diagrams
\[ \begin{tikzcd}
    X\ar[r]\ar[d]\ar[dr,phantom,pos=.8,"\lrcorner"] & *\ar[d] \\
    *\ar[r] & \Sigma X
\end{tikzcd}
\quad\text{and}\quad
\begin{tikzcd}
    \Omega X\ar[r]\ar[d]\ar[dr,phantom,pos=.2,"\ulcorner"] & *\ar[d] \\
    *\ar[r] & X\rlap{\ ,}
\end{tikzcd} \]
as has been shown in the last section.
In a stable category, we instead have the pushout-pullback diagram
\[ \begin{tikzcd}
    X\ar[r]\ar[d]\ar[dr,phantom,start anchor=south east,"\ulcorner" pos=.25,"\lrcorner" pos=.85] & *\ar[d] \\
    *\ar[r] & X[1]\rlap{\ ,}
\end{tikzcd} \]
which means that $\Sigma$ and $\Omega$ are inverse to each other.
For example, we will see that for cochain complexes,
the functors $\Sigma$ and $\Omega$ coincide with the shifting functors
$[1]$ and $[-1]$, and thus, the category of cochain complexes
is an example of a stable category.

For any category with finite limits, we will describe a \emph{stabilisation}
procedure, that turns the category into a stable one.
For example, the stabilisation of the category of topological spaces
will be the category of spectra, which we will define below.

From now on, when referring to quasi-categories,
we will change our terminology to ``\term{$(\infty,1)$-categories}'',
or ``\term{$\infty$-categories}'' for short.
This means that we are not only studying one particular model;
we are also talking about the behaviour of an intrinsic notion of $(\infty,1)$-categories,
which is independent of models.
However, we only provide rigorous formulations for quasi-categories.

\subsection{Stable categories}

\begin{definition}
    An $\infty$-category $\cat C$ is said to be \term{pointed},
    if it has a zero object $0\in\cat C$,
    i.e.\ an object that is both initial and terminal.
\end{definition}

\begin{definition}
    Let $\cat C$ be a pointed $\infty$-category.
    \begin{itms}
        \item A \term{triangle} in $\cat C$ is a diagram 
        \[\begin{tikzcd}
            X\ar[d]\ar[r] & Y\ar[d] \\
            0\ar[r] & Z\rlap{\ .}
        \end{tikzcd}\]
        
        \item A \term{cofibre sequence} in $\cat C$
        is a triangle that is also a pushout square.
        In this case, we say that $Z$ is the
        \term{homotopy cofibre} of the map $X\to Y$.
        
        \item A \term{fibre sequence} in $\cat C$
        is a triangle that is also a pullback square.
        In this case, we say that $X$ is the
        \term{homotopy fibre} of the map $Y\to Z$.
    \end{itms}
\end{definition}

\begin{definition}
    A \term{stable $\infty$-category} is a pointed $\infty$-category $\cat C$,
    such that
    \begin{itms}
        \item Every morphism in $\cat C$ has a homotopy cofibre and a homotopy fibre.
        \item A triangle in $\cat C$ is a cofibre sequence
        if and only if it is a fibre sequence.
    \end{itms}
\end{definition}

\begin{definition}
    Let $\cat C$ be a pointed $\infty$-category.
    The functors $\Sigma\:\cat C\to\cat C$ and $\Omega\:\cat C\to\cat C$
    are defined by 
    \[ \Sigma X:=0\underset{X}{\sqcup}0
    \quad\text{and}\quad
    \Omega X:=0\underset{X}{\times}0. \]
\end{definition}

As one might expect, $\Sigma$ and $\Omega$ are adjoint to each other, 
since for any $X,Y\in\cat C$, one has 
\[ \Hom_{\cat C}(\Sigma X,Y)\simeq
\{*\}\underset{\Hom_{\cat C}(X,Y)}{\times}\{*\}
\simeq\Hom_{\cat C}(X,\Omega Y). \]

If $\cat C$ is stable, then $\Sigma$ and $\Omega$ are inverse to each other.
In this case, we denote 
\[ X[1]:=\Sigma X\quad\text{and}\quad X[-1]:=\Omega X. \]
For a non-negative integer $n$, we denote
\[ X[n]:=\Sigma^nX\quad\text{and}\quad X[-n]:=\Omega^nX. \]

\begin{proposition}
    Let $\cat C$ be a pointed $\infty$-category 
    that admits pushouts and pullbacks. Then the following are equivalent.
    \begin{itms}
        \item $\Sigma$ is an equivalence.
        \item $\Omega$ is an equivalence.
        \item $\cat C$ is stable.
        \item A square in $\cat C$ is a pushout square,
        if and only if it is a pullback square.
    \end{itms}
\end{proposition}

\begin{proof}
    \nyw
\end{proof}

\begin{definition}
    Let $\cat C,\cat D$ be stable $\infty$-categories.
    A functor $f\:\cat C\to\cat D$ is said to be \term{exact},
    if it preserves the zero object,
    and preserves (co)fibre sequences.
\end{definition}

\subsection{Homotopy category of a stable category}

Our goal in this subsection is to prove the following theorem.

\begin{theorem}
    Let $\cat C$ be a stable $\infty$-category.
    Then the homotopy category $\Ho(\cat C)$
    carries a natural structure of a triangulated category.
\end{theorem}

\begin{proof}
    First, we need to show that $\Ho(\cat C)$ is an additive category.

    \begin{itms}
        \item \emph{$\Ho(\cat C)$ has finite coproducts.}
        
        By (\ref{thm-6-z}), a coproduct in $\cat C$ gives rise to a coproduct in $\Ho(\cat C)$.
        By definition, $\cat C$ has an initial object.
        Thus it suffices to show that for any $X,Y\in\cat C$, the coproduct $X\sqcup Y$ exists.
        We form the pushout diagram 
        \[ \begin{tikzcd}
            X[-1]\ar[r,"0"]\ar[d]\ar[dr,phantom,"\ulcorner" pos=.15,"\lrcorner" pos=.75] & Y\ar[d] \\
            0\ar[r] & Z
        \end{tikzcd} \]
        in $\cat C$. Then $Z$ is the coproduct $X\sqcup Y$, since
        \[ \begin{aligned}
            Z &\simeq \operatorname{cofibre}(X[-1]\xrightarrow0Y) \\
            &\simeq \operatorname{cofibre}(X[-1]\to0)\sqcup\operatorname{cofibre}(0\to Y) \\
            &\simeq X\sqcup Y,
        \end{aligned} \]
        where the second step uses the fact that taking the cofibre 
        commutes with colimits,
        as can be checked by the definition of homotopy colimits.

        \item \emph{$\Ho(\cat C)$ is an additive category.}
        
        For any $X,Y\in\cat{C}$, note that 
        \[\begin{aligned}
            \Hom_{\cat{C}}(X[1],Y)
            &\simeq \Hom_{\cat{C}}(0,Y) \underset{\Hom_{\cat{C}}(X,Y)}{\times} \Hom_{\cat{C}}(0,Y) \\
            &\simeq \Omega\Hom_{\cat{C}}(X,Y).
        \end{aligned}\]
        Therefore,
        \[
            \pi_0\Hom_{\cat{C}}(X,Y)\simeq 
            \pi_1\Hom_{\cat{C}}(X[-1],Y)\simeq 
            \pi_2\Hom_{\cat{C}}(X[-2],Y)\simeq\cdots
        \]
        is an abelian group.
        We leave it to the reader to check that 
        composition is bilinear.

        \item \emph{For a map $f\:X\to Y$, what is $-f$?}
        
        A map from $X$ to $Y$ can be seen as a diagram
        \[\begin{tikzcd}
            X[-1]\ar[r]\ar[d] & 0\ar[d] \\
            0\ar[r] & Y\rlap{\ .}
        \end{tikzcd}\]
        If we swap the two $0$'s, the map will reverse its sign.
        Precisely, if this square corresponds to the map $f$,
        then the transpose of this square will correspond to $-f$.
        This is because the group structure is defined by the identification
        \[\Hom_{\cat{C}}(X,Y)\simeq\{*\}\underset{\Hom_{\cat{C}}(X[-1],Y)}{\times}\{*\}
        \simeq\Omega\Hom_{\cat{C}}(X[-1],Y),\]
        and if the two $\{*\}$'s are swapped,
        then the loop space will have its loops reversed.
    \end{itms}

    Next, we define a triangulated structure on $\Ho(\cat{C})$,
    and we verify the axioms of a triangulated category.

    \begin{itms}
        \item \emph{Distinguished triangles.}
        
        We define a triangle $X\to Y\to Z\to X[1]$ in $\Ho(\cat{C})$
        to be a distinguished triangle,
        if it is isomorphic to one that can be lifted to a diagram 
        \[\begin{tikzcd}
            X\ar[r]\ar[d]\ar[dr,phantom,"\ulcorner" pos=.25,"\lrcorner" pos=.75] &
            Y\ar[r]\ar[d]\ar[dr,phantom,"\ulcorner" pos=.25,"\lrcorner" pos=.75] & 0\ar[d] \\
            0\ar[r] & Z\ar[r] & W
        \end{tikzcd}\]
        in $\cat{C}$, where $W$ is equivalent to $X[-1]$
        since the outer square is a pushout-pullback.

        \item (TR 1)
        \begin{itms}
            \item \emph{Distinguished triangles are stable under isomorphisms.}
            \item \emph{For any object $X$, the following triangle is distinguished:
            \[X\xrightarrow{1_X}X\to0\to X[1].\]}
            \item \emph{Every map $X\to Y$ extends to a distinguished triangle \[X\to Y\to Z\to X[1].\]}
        \end{itms}

        These properties are all obvious.

        \item (TR 2)
        \emph{If
        \[X\xrightarrow{f}Y\xrightarrow{g}Z\xrightarrow{h}X[1]\]
        is a distinguished triangle, so is
        \[Y\xrightarrow{g}Z\xrightarrow{h}X[1]\xrightarrow{-f[1]}Y[1].\]}%

        To prove this, let us form the diagram 
        \[\begin{tikzcd}
            X\ar[r]\ar[d]\ar[dr,phantom,"\ulcorner" pos=.25,"\lrcorner" pos=.75] &
            Y\ar[r]\ar[d]\ar[dr,phantom,"\ulcorner" pos=.25,"\lrcorner" pos=.75] & 0\ar[d] \\
            0\ar[r] &
            Z\ar[r]\ar[d]\ar[dr,phantom,"\ulcorner" pos=.25,"\lrcorner" pos=.75] &
            W\ar[d] \\
            & 0\ar[r] & V
        \end{tikzcd}\]
        in $\cat{C}$.
        The two outer rectangles establish equivalences $W\simeq X[1]$ and $V\simeq Y[1]$.
        The map $W\to V$ in the diagram is $f[1]$,
        since it is induced by the map of diagrams
        \[\begin{tikzcd}
            X\ar[r]\ar[d] & 0 \\ 0
        \end{tikzcd}
        \quad\longrightarrow\quad
        \begin{tikzcd}
            Y\ar[r]\ar[d] & 0 \\ 0
        \end{tikzcd}\]
        in $\cat C$. However, after transposing the diagram 
        to match the definition of distinguished triangles,
        the map $W\to V$ becomes $-f[1]$.

        \item (TR 3)
        \emph{Given a diagram}
        \[\begin{tikzcd}
            X\ar[r]\ar[d,"f"'] & Y\ar[r]\ar[d] & Z\ar[r]\ar[d,dashed] & X[1]\ar[d,"f\lbrack1\rbrack"] \\
            X'\ar[r] & Y'\ar[r] & Z'\ar[r] & X'[1]
        \end{tikzcd}\]
        \emph{without the dashed arrow, where the two rows are distinguished triangles,
        there exists a dashed arrow making the diagram commute.}

        This is because taking cofibres in $\cat{C}$ is functorial.

        \item (TR 4)
        \emph{The octahedral axiom.}

        We omit the proof here, but it is not difficult.
    \end{itms}

    We conclude that $\Ho(\cat{C})$
    is a triangulated category.
\end{proof}

\subsection{Spectra and stable homotopy theory}

In this section, we review the classical theory of
topological spectra and stable homotopy theory,
as a model for the stabilisation procedure that will
apply to general stable $\infty$-categories.

Stable homotopy theory is concerned with properties of a topological space 
that stabilise after applying the suspension functor sufficiently many times.
For example, the \term{stable homotopy groups} of a pointed topological space~$X$
are defined by 
\[\pi_n^{\mathrm s}(X):=
\mathop{\operatorname{colim}}_{k\to+\infty} \pi_{n+k}(\Sigma^kX)\]
for all $n\in\mathbb{Z}$.
For example, the Freudenthal suspension theorem states that 
\[\pi_n^{\mathrm s}(S^0)\simeq\pi_{2n+2}(S^{n+2})
\simeq\pi_{2n+3}(S^{n+3})\simeq\cdots.\]
Topological spectra contain the data that encodes
these stabilised properties of a topological space.

\begin{definition}
    A \term{spectrum}~$E$ consists of
    \begin{itms}
        \item A series of pointed topological spaces 
        \[E_0,E_1,E_2,\dotsc.\]
        \item For each $n\geq0$, a map 
        \[\sigma_n\:\Sigma E_n\to E_{n+1}.\]
    \end{itms}
    A \term{map of spectra}~$f$ between two spectra $E$~and~$F$
    consists of a map
    \[f_n\:E_n\to F_n\]
    for each $n\geq0$,
    such that $\sigma_n\circ\Sigma f_n=f_{n+1}\circ\sigma_n$ for all $n\geq0$.
\end{definition}

The category of spectra is denoted by~$\cat{Sp}$.

\begin{example}
    Let~$X$ be a pointed topological space.
    The \term{suspension spectrum}~$\Sigma^\infty X$ is defined by 
    \[ (\Sigma^\infty X)_n:=\Sigma^nX, \]
    with $\sigma_n:=1_{\Sigma^{n+1}X}$.
    The purpose of the notation~$\Sigma^\infty$ is to signify that 
    we are looking for the properties of~$\Sigma^nX$ as $n\to\infty$.

    The \term{sphere spectrum}~$\mathbb{S}$ is defined by
    \[ \mathbb{S}:=\Sigma^\infty S^0, \]
    so that $\mathbb{S}_n\simeq S^n$. \varqed
\end{example}

\begin{definition}
    The \term{stable homotopy groups} of a spectrum $E$ are defined by 
    \[\pi_n^{\mathrm s}(E):=
    \mathop{\operatorname{colim}}_{k\to+\infty} \pi_n(E_k)\]
    for all $n\in\mathbb{Z}$.
\end{definition}

For example, we have 
\[ \pi_n(\Sigma^\infty X)\simeq\pi_n^{\mathrm s}(X) \]
for any pointed topological space $X$.

Note that the suspension functor~$\Sigma$
and the shifting functor~$[1]$
induce the same maps on stable homotopy groups.
We will see that they are equivalent up to homotopy,
given a certain model structure on~$\cat{Sp}$.

\begin{definition}
    A spectrum~$E$ is called an \term{$\Omega$-spectrum},
    if the maps 
    \[E_n\to\Omega E_{n+1}\]
    corresponding to $\sigma_n$
    are weak equivalences for all $n\geq0$.
\end{definition}

In this case, we write $\Omega^\infty E:=E_0$.
For any $n\geq0$, we have $\pi_n(E)\simeq\pi_n(E_0)$.

\begin{theorem}
    The category~$\cat{Sp}$ has the \term{stable model structure}, with 
    \begin{itms}
        \item $\cat{W}=\{$maps that induce isomorphisms of all
        stable homotopy groups$\}$.
        \item $\cat{Cof}=\{$degreewise cofibrations$\}$.
        \item The fibrant objects are precisely the $\Omega$-spectra.
    \end{itms}
    Moreover, the adjoint pair 
    \[\begin{tikzcd}
        \cat{Sp}
        \ar[r,bend left,"\Sigma"]
        \ar[r,phantom,"\bot"] &
        \cat{Sp}
        \ar[l,bend left,"\Omega"]
    \end{tikzcd}\]
    is a Quillen equivalence, and their derived functors
    are isomorphic to the shifting functors:
    \[ \mathbb{L}\Sigma\simeq[1]
    \quad\text{and}\quad
    \mathbb{R}\Omega\simeq[-1]. \]
\end{theorem}

\begin{corollary}
    The $\infty$-category of spectra,
    presented by the model category of spectra,
    is a stable $\infty$-category.
\end{corollary}

\begin{proof}
    It suffices to check the existence of homotopy (co)fibres.
    However, by definition, every model category admits arbitrary (co)limits.
\end{proof}

As a result, the homotopy category $\Ho(\cat{Sp})$
has a triangulated structure.

Spectra can also be viewed from another perspective,
namely, as representing objects of generalised cohomology theories.

\begin{definition}
    A \term{generalised cohomology theory}
    is a family of functors
    \[ \widetilde E^n\:\cat{Top}\op_*\to\cat{Ab}, \]
    where $n\in\mathbb Z$,
    and $\cat{Top}_*$ denotes the category of pointed topological spaces,
    satisfying the following axioms.
    \begin{itms}
        \item \textup{(Homotopy)} If $f\:X\to Y$ is a weak homotopy equivalence,
        then $\widetilde E^\bullet(f)$ is an isomorphism.
        \item \textup{(Exactness)} A homotopy cofibre sequence $X\to Y\to Z$ induces an exact sequence
        \[ \widetilde E^\bullet(Z)\to\widetilde E^\bullet(Y)\to\widetilde E^\bullet(X). \]
        \item \textup{(Suspension)} There is a natural isomorphism
        \[ \widetilde E^{\bullet+1}(\Sigma X)\simeq\widetilde E^\bullet(X). \]
        \item \textup{(Additivity)} $\widetilde E^\bullet$ takes coproducts to products:
        \[\textstyle \widetilde E^\bullet\bigl(\bigvee_\alpha X_\alpha\bigr)\simeq
        \prod_\alpha\widetilde E^\bullet(X_\alpha). \]
    \end{itms}
\end{definition}

For example, reduced singular cohomology with coefficients in any abelian group 
is a generalised cohomology theory.
$K$-theory is also a generalised cohomology theory.

\begin{theorem}[Brown]
    Every generalised cohomology theory $\widetilde{E}^\bullet$
    is represented by a spectrum $E$, in the sense that 
    \[\widetilde{E}^n(X)\simeq\Hom_{\Ho(\cat{Sp})}(\Sigma^\infty X[-n],E)\]
    for any topological space $X$.
\end{theorem}

Therefore, the category of spectra,
which we regard as a ``stabilisation'' of the category of topological spaces,
is equivalent (in the homotopical sense)
to the category of generalised cohomology theories of topological spaces.

\subsection{Stabilisation}

Let $\cat C$ be an $\infty$-category with finite limits.
The goal of this subsection is to construct
a stable $\infty$-category $\operatorname{Sp}(\cat{C})$ of \emph{spectrum objects} of $\cat C$,
together with a functor
\[\Omega^\infty\:\operatorname{Sp}(\cat{C})\to\cat{C},\]
which is universal in the sense that for any stable $\infty$-category $\cat D$,
every functor $\cat{D}\to\cat{C}$ preserving finite limits 
will factor through the functor $\Omega^\infty$ uniquely.

Before we give the general construction, 
let us introduce another perspective
from which topological spectra can be viewed.

Let $E$ be an $\Omega$-spectrum.
Let $\cat{Top}_*^{\mathrm{fin}}$ be the category 
of pointed topological spaces which are homotopy equivalent
to finite CW~complexes.
Then we may define a functor 
\[F\:\cat{Top}_*^{\mathrm{fin}}\to\cat{Top}_*,\]
sending $S^n$ to the space~$E_n$.
To define such a functor,
notice that such a functor sends a pushout square to a pullback square:
\[\begin{tikzcd}
    S^n\ar[r]\ar[d]\ar[dr,phantom,pos=.75,"\lrcorner"] & 0\ar[d] \\
    0\ar[r] & S^{n+1}
\end{tikzcd}
\quad\overset{F}{\longmapsto}\quad
\begin{tikzcd}
    E_n\ar[r]\ar[d]\ar[dr,phantom,pos=.25,"\ulcorner"] & 0\ar[d] \\
    0\ar[r] & E_{n+1}\rlap{\ .}
\end{tikzcd}\]
If we require that $F$ sends any pushout square to a pullback square,
then $F$ will be uniquely defined (up to an equivalence),
assuming that the spaces~$E_n$ can be delooped arbitrarily many times,
which is the case since we assumed that $E$ is an $\Omega$-spectrum.
(The reader may prove this as an exercise.)

The requirement that $F$ should send a pushout square to a pullback square
is analogous to the excision property of classical homology theories.
Therefore, spectra may be seen as ``homology theories'' of topological spaces,
with values in topological spaces.

In general, we will define a \emph{spectrum object} of $\cat{C}$
to be a ``homology theory'' of topological spaces,
with values in $\cat{C}$.

\begin{definition}
    Let $\cat{C}$ and $\cat{D}$ be $\infty$-categories.
    \begin{itms}
        \item Suppose $\cat{C}$ has pushouts.
        A functor $F\:\cat{C}\to\cat{D}$ is \term{excisive}
        if it sends pushout squares to pullback squares.
        \item Suppose $\cat{C}$ has a zero object.
        A functor $F\:\cat{C}\to\cat{D}$ is \term{reduced}
        if $F$ preserves the zero object.
    \end{itms}
    We denote by 
    \[\operatorname{Fun}_*(\cat{C},\cat{D})\quad\text{and}\quad
    \operatorname{Exc}_*(\cat{C},\cat{D})\]
    the categories of reduced functors and reduced excisive functors 
    from $\cat{C}$ to $\cat{D}$, respectively,
    as full subcategories of $\operatorname{Fun}(\cat{C},\cat{D})$.
\end{definition}

Recall that the category of spaces was defined to be
$\cat{S}:=\mathfrak{N}(\cat{Kan})$.

\begin{definition}
    The category of \term{finite spaces}, denoted by $\cat{S}^{\mathrm{fin}}$,
    is defined to be the full subcategory of $\cat{S}$
    consisting of Kan complexes whose geometric realisations
    are equivalent to finite CW complexes. Let
    \[\cat{S}_*^{\mathrm{fin}}:=(\cat{S}^{\mathrm{fin}})_{\{*\}\/}\]
    be the category of pointed finite spaces.
\end{definition}

\begin{definition}
    Let $\cat{C}$ be an $\infty$-category.
    The category of \term{spectrum objects} of $\cat{C}$
    is defined by 
    \[\operatorname{Sp}(\cat{C}):=
    \operatorname{Exc}_*(\cat{S}_*^{\mathrm{fin}},\cat{C}).\]
    The functor 
    \[\Omega^\infty\:\operatorname{Sp}(\cat{C})\to\cat{C}\]
    is defined by evaluating at $S^0\in\cat{S}_*^{\mathrm{fin}}$.
\end{definition}

For example, the above discussion suggests that we should have
\[\cat{Sp}\simeq\operatorname{Sp}(\cat{S}_*)\simeq\operatorname{Sp}(\cat{Top}_*).\]

\begin{proposition}
    The category $\operatorname{Sp}(\cat{C})$ is stable
    if $\cat{C}$ admits finite limits.

    More generally, the category $\operatorname{Exc}_*(\cat{C},\cat{D})$
    is stable if $\cat{C}$ admits finite colimits and 
    $\cat{D}$ admits finite limits.
\end{proposition}

\begin{proof}
    \nyw
\end{proof}

\begin{proposition}
    Let $\cat{C}$ be an $\infty$-category with finite limits.
    Then $\cat{C}$ is stable if and only if the functor 
    $\Omega^\infty\:\operatorname{Sp}(\cat{C})\to\cat{C}$
    is an equivalence.
\end{proposition}

\begin{proof}
    \nyw
\end{proof}

\begin{proposition}
    Let $\cat{C}$ and $\cat{D}$ be $\infty$-categories.

    \begin{itms}
        \item 
        If $\cat{C}$ is stable and $\cat{D}$ has finite limits, then
        \[\operatorname{Fun}^{\mathrm{l.ex.}}(\cat{C},\operatorname{Sp}(\cat{D}))
        \simeq\operatorname{Fun}^{\mathrm{l.ex.}}(\cat{C},\cat{D})\]
        \item 
        More generally, if $\cat{C}$ is pointed and has finite limits,
        and $\cat{D}$ has finite limits, then 
    \end{itms}
\end{proposition}

\tbc
