In topology, we have homotopy pushout and pullback diagrams
\[ \begin{tikzcd}
    X\ar[r]\ar[d]\ar[dr,phantom,pos=.8,"\lrcorner"] & *\ar[d] \\
    *\ar[r] & \Sigma X
\end{tikzcd}
\quad\text{and}\quad
\begin{tikzcd}
    \Omega X\ar[r]\ar[d]\ar[dr,phantom,pos=.2,"\ulcorner"] & *\ar[d] \\
    *\ar[r] & X\rlap{\ ,}
\end{tikzcd} \]
as has been shown in the last section.
In a stable category, we instead have the pushout-pullback diagram
\[ \begin{tikzcd}
    X\ar[r]\ar[d]\ar[dr,phantom,start anchor=south east,"\ulcorner" pos=.25,"\lrcorner" pos=.85] & *\ar[d] \\
    *\ar[r] & X[1]\rlap{\ ,}
\end{tikzcd} \]
which means that $\Sigma$ and $\Omega$ are inverse to each other.
For example, we will see that for cochain complexes,
the functors $\Sigma$ and $\Omega$ coincide with the shifting functors
$[1]$ and $[-1]$, and thus, the category of cochain complexes
is an example of a stable category.

For any category with finite limits, we will describe a \emph{stabilisation}
procedure, that turns the category into a stable one.
For example, the stabilisation of the category of topological spaces
will be the category of spectra, which we will define below.

From now on, when referring to quasi-categories,
we will change our terminology to ``\term{$(\infty,1)$-categories}'',
or ``\term{$\infty$-categories}'' for short.
This means that we are not only studying one particular model;
we are also talking about the behaviour of an intrinsic notion of $(\infty,1)$-categories,
which is independent of models.
However, we only provide rigorous formulations for quasi-categories.

\subsection{Definition and examples}

\nyw

\subsection{Homotopy category}

\nyw

\subsection{Spectra and stable homotopy theory}

\nyw

\subsection{Stabilisation}

\nyw
