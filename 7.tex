In topology, we have homotopy pushout and pullback diagrams
\[ \begin{tikzcd}
    X\ar[r]\ar[d]\ar[dr,phantom,pos=.8,"\lrcorner"] & *\ar[d] \\
    *\ar[r] & \Sigma X
\end{tikzcd}
\quad\text{and}\quad
\begin{tikzcd}
    \Omega X\ar[r]\ar[d]\ar[dr,phantom,pos=.2,"\ulcorner"] & *\ar[d] \\
    *\ar[r] & X\rlap{\ ,}
\end{tikzcd} \]
as has been shown in the last section.
In a stable category, we instead have the pushout-pullback diagram
\[ \begin{tikzcd}
    X\ar[r]\ar[d]\ar[dr,phantom,start anchor=south east,"\ulcorner" pos=.25,"\lrcorner" pos=.85] & *\ar[d] \\
    *\ar[r] & X[1]\rlap{\ ,}
\end{tikzcd} \]
which means that $\Sigma$ and $\Omega$ are inverse to each other.
For example, we will see that for cochain complexes,
the functors $\Sigma$ and $\Omega$ coincide with the shifting functors
$[1]$ and $[-1]$, and thus, the category of cochain complexes
is an example of a stable category.

For any category with finite limits, we will describe a \emph{stabilisation}
procedure, that turns the category into a stable one.
For example, the stabilisation of the category of topological spaces
will be the category of spectra, which we will define below.

From now on, when referring to quasi-categories,
we will change our terminology to ``\term{$(\infty,1)$-categories}'',
or ``\term{$\infty$-categories}'' for short.
This means that we are not only studying one particular model;
we are also talking about the behaviour of an intrinsic notion of $(\infty,1)$-categories,
which is independent of models.
However, we only provide rigorous formulations for quasi-categories.

\subsection{Stable categories}

\begin{definition}
    An $\infty$-category $\cat C$ is said to be \term{pointed},
    if it has a zero object $0\in\cat C$,
    i.e.\ an object that is both initial and terminal.
\end{definition}

\begin{definition}
    Let $\cat C$ be a pointed $\infty$-category.
    \begin{itms}
        \item A \term{triangle} in $\cat C$ is a diagram 
        \[\begin{tikzcd}
            X\ar[d]\ar[r] & Y\ar[d] \\
            0\ar[r] & Z\rlap{\ .}
        \end{tikzcd}\]
        
        \item A \term{cofibre sequence} in $\cat C$
        is a triangle that is also a pushout square.
        In this case, we say that $Z$ is the
        \term{homotopy cofibre} of the map $X\to Y$.
        
        \item A \term{fibre sequence} in $\cat C$
        is a triangle that is also a pullback square.
        In this case, we say that $X$ is the
        \term{homotopy fibre} of the map $Y\to Z$.
    \end{itms}
\end{definition}

\begin{definition}
    A \term{stable $\infty$-category} is a pointed $\infty$-category $\cat C$,
    such that
    \begin{itms}
        \item Every morphism in $\cat C$ has a homotopy cofibre and a homotopy fibre.
        \item A triangle in $\cat C$ is a cofibre sequence
        if and only if it is a fibre sequence.
    \end{itms}
\end{definition}

\begin{definition}
    Let $\cat C$ be a pointed $\infty$-category.
    The functors $\Sigma\:\cat C\to\cat C$ and $\Omega\:\cat C\to\cat C$
    are defined by 
    \[ \Sigma X:=0\underset{X}{\sqcup}0
    \quad\text{and}\quad
    \Omega X:=0\underset{X}{\times}0. \]
\end{definition}

As one might expect, $\Sigma$ and $\Omega$ are adjoint to each other, 
since for any $X,Y\in\cat C$, one has 
\[ \Hom_{\cat C}(\Sigma X,Y)\simeq
\{*\}\underset{\Hom_{\cat C}(X,Y)}{\times}\{*\}
\simeq\Hom_{\cat C}(X,\Omega Y). \]

If $\cat C$ is stable, then $\Sigma$ and $\Omega$ are inverse to each other.
In this case, we denote 
\[ X[1]:=\Sigma X\quad\text{and}\quad X[-1]:=\Omega X. \]
For a non-negative integer $n$, we denote
\[ X[n]:=\Sigma^nX\quad\text{and}\quad X[-n]:=\Omega^nX. \]

\begin{proposition}
    Let $\cat C$ be a pointed $\infty$-category 
    that admits pushouts and pullbacks. Then the following are equivalent.
    \begin{itms}
        \item $\Sigma$ is an equivalence.
        \item $\Omega$ is an equivalence.
        \item $\cat C$ is stable.
        \item A square in $\cat C$ is a pushout square,
        if and only if it is a pullback square.
    \end{itms}
\end{proposition}

\begin{proof}
    \nyw
\end{proof}

\begin{definition}
    Let $\cat C,\cat D$ be stable $\infty$-categories.
    A functor $f\:\cat C\to\cat D$ is said to be \term{exact},
    if it preserves the zero object,
    and preserves (co)fibre sequences.
\end{definition}

\subsection{Homotopy category of a stable category}

Our goal in this subsection is to prove the following theorem.

\begin{theorem}
    Let $\cat C$ be a stable $\infty$-category.
    Then the homotopy category $\Ho(\cat C)$
    carries a natural structure of a triangulated category.
\end{theorem}

\begin{proof}
    First, we need to show that $\Ho(\cat C)$ is an additive category.

    \begin{itms}
        \item \emph{$\Ho(\cat C)$ has finite coproducts.}
        
        By (\ref{thm-6-z}), a coproduct in $\cat C$ gives rise to a coproduct in $\Ho(\cat C)$.
        By definition, $\cat C$ has an initial object.
        Thus it suffices to show that for any $X,Y\in\cat C$, the coproduct $X\sqcup Y$ exists.
        We form the pushout diagram 
        \[ \begin{tikzcd}
            X[-1]\ar[r,"0"]\ar[d]\ar[dr,phantom,"\ulcorner" pos=.15,"\lrcorner" pos=.75] & Y\ar[d] \\
            0\ar[r] & Z
        \end{tikzcd} \]
        in $\cat C$. Then $Z$ is the coproduct $X\sqcup Y$, since
        \[ \begin{aligned}
            Z &\simeq \operatorname{cofibre}(X[-1]\xrightarrow0Y) \\
            &\simeq \operatorname{cofibre}(X[-1]\to0)\sqcup\operatorname{cofibre}(0\to Y) \\
            &\simeq X\sqcup Y,
        \end{aligned} \]
        where the second step uses the fact that taking the cofibre 
        commutes with colimits,
        as can be checked by the definition of homotopy colimits.
    \end{itms}
\end{proof}

\tbc

\subsection{Spectra and stable homotopy theory}

\nyw

\subsection{Stabilisation}

\nyw
