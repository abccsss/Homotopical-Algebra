Assume we have a functor $f\:X\to S$ between two categories.
Then for any $s\in S$, we may take the fibre $X_s:=f^{-1}(s)$,
which is a subcategory of $X$.
This gives a rule assigning every object of $S$ a category.
We might expect that this assignment, i.e.\ taking the fibres,
gives a functor $S\to\cat{Cat}$.
Indeed, this is true provided that $f$ is a ``fibration'',
in a sense that will be made precise soon.
This is called the \term{Grothendieck construction}.
Schematically, this means that we have a correspondence
\[\begin{tikzcd}[row sep=1ex,column sep=1em]
    X\ar[dd,"\text{fibration}"'] && \cat{Cat} \\
    & \Longleftrightarrow \\
    S && S\ar[uu,"\text{take fibres}"']
\end{tikzcd}\]
between ``fibrations'' over $S$ and functors from $S$ to $\cat{Cat}$.

Let us look at an example arising from algebraic geometry.

\begin{example}\label{eg-5-q}
    Let $\cat{Sch}$ denote the category of schemes.
    For a scheme $X$, let $\cat{QCoh}_X$ be the category 
    of quasi-coherent sheaves on $X$.
    This defines a functor 
    \[ \cat{QCoh}_{(-)}\:\cat{Sch}\op\to\cat{Cat}. \]

    If we look carefully, it is not a functor in the usual sense,
    since the composition law 
    \[ f^*\circ g^*=(g\circ f)^* \]
    only holds up to a natural isomorphism, i.e.\ a $2$-morphism in $\cat{Cat}$.
    We call such a ``functor'' a \term{$2$-functor}.

    Applying the Grothendieck construction to this $2$-functor,
    we should get a ``fibration'' which we denote by
    \[ \cat{QCoh}\to\cat{Sch}\op, \]
    whose fibre over $X\in\cat{Sch}\op$ is the category $\cat{QCoh}_X$.
    Indeed, we may construct the category $\cat{QCoh}$ of all quasi-coherent sheaves as follows.
    \begin{itms}
        \item The objects are pairs $(X,\mathscr F)$,
        where $\mathscr F$ is a quasi-coherent sheaf on the scheme $X$.
        \item A morphism from $(X,\mathscr F)$ to $(Y,\mathscr G)$
        consists of a map of schemes $f\:Y\to X$, together with
        a map of $\mathscr O_Y$-modules $f^*\mathscr F\to\mathscr G$, or equivalently,
        a map of $\mathscr O_X$-modules $\mathscr F\to f_*\mathscr G$.
    \end{itms}
    The natural forgetful functor $\cat{QCoh}\to\cat{Sch}\op$
    is the ``fibration'' that we wished to construct.
    This is an example of a \emph{cocartesian fibration},
    which we will define later in this section. \varqed
\end{example}

\subsection{For ordinary categories}

We will study four kinds of fibrations of categories,
which correspond to functors to $\cat{Cat}$
according to the following table,
where $\cat{Gpd}$ denotes the category of groupoids.

\begin{center}
    \begin{tabular}{ccc}
        \textbf{Fibration} & & \textbf{Functor} \\ \hline
        left fibration $X\to S$ & $\Longleftrightarrow$ & $S\to\cat{Gpd}$ \\
        right fibration $X\to S$ & $\Longleftrightarrow$ & $S\op\to\cat{Gpd}$ \\
        cocartesian fibration $X\to S$ & $\Longleftrightarrow$ & $S\to\cat{Cat}$ \\
        cartesian fibration $X\to S$ & $\Longleftrightarrow$ & $S\op\to\cat{Cat}$ \\
    \end{tabular}
\end{center}

We start with left fibrations,
which Grothendieck originally called ``categories cofibred in groupoids''.
By the table above, for any $s\in S$,
the fibre $X_s$ should be a groupoid,
and for any morphism $s\to s'$ in $S$,
we should have a ``transport map'' $X_s\to X_{s'}$.

In order to define the transport map, 
we require that the map $p\:X\to S$ satisfies the following properties.
\begin{itms}
    \item Transport of objects: for any $x\in X$ and any morphism $\alpha\:s\to s'$ in $S$,
    where $s:=p(x)$,
    there exists a morphism $\tilde\alpha\:x\to x'$ in $X$ such that $p(\tilde\alpha)=\alpha$.
    Pictorially, this means that the lifting problem
    \[\begin{tikzcd}
        x\ar[d,dotted,dash] \\
        s \ar[r] & s'
    \end{tikzcd}\quad\Longrightarrow\quad\begin{tikzcd}
        x\ar[r,dashed]\ar[d,dotted,dash] & x'\ar[d,dotted,dash] \\
        s \ar[r] & s'
    \end{tikzcd}\]
    % \[\vcenter{\xymatrix @R=2em { x\ar@{.}[d] \\ s \ar[r] & s' }}\quad\Longrightarrow\quad\vcenter{\xymatrix @R=2em { x\ar@{-->}[r]\ar@{.}[d] & x'\ar@{.}[d] \\ s \ar[r] & s' }}\]
    has a solution.
    \item Transport of morphisms: the lifting problem 
    \[\begin{tikzcd}
        x\ar[d]\ar[r] & x' \\
        y\ar[d,dotted,dash]\ar[r] & y'\ar[d,dotted,dash] \\
        s \ar[r] & s'
    \end{tikzcd}\quad\Longrightarrow\quad\begin{tikzcd}
        x\ar[d]\ar[r] & x'\ar[d,dashed] \\
        y\ar[d,dotted,dash]\ar[r] & y'\ar[d,dotted,dash] \\
        s \ar[r] & s'
    \end{tikzcd}\]
    % \[\vcenter{\xymatrix @R=2em { x\ar[d]\ar[r] & x' \\ y\ar@{.}[d]\ar[r] & y'\ar@{.}[d] \\ s \ar[r] & s' }}\quad\Longrightarrow\quad\vcenter{\xymatrix @R=2em { x\ar[d]\ar[r] & x'\ar@{-->}[d] \\ y\ar@{.}[d]\ar[r] & y'\ar@{.}[d] \\ s \ar[r] & s' }}\]
    has a \emph{unique} solution.
\end{itms}

It is easy to see that if these properties are satisfied,
then the transport map $X_s\to X_{s'}$ is well-defined up to a natural isomorphism.
Namely, one chooses an arbitrary way to transport the objects,
and then the morphisms can be transported uniquely.

We define left fibrations by reformulating these axioms.

\begin{definition}
    A functor $p\:X\to S$ is a \term{left fibration}, if
    it satisfies the following transport axioms.
    \begin{itms}
        \item The lifting problem 
        \[\begin{tikzcd}
            \Lambda_0[1] \ar[d] \ar[r] & X \ar[d] \\
            \Delta[1] \ar[r] \ar[ur,dashed] & S
        \end{tikzcd}\]
        % \[\xymatrix{ \Lambda_0[1] \ar[d] \ar[r] & X \ar[d] \\ \Delta[1] \ar[r] \ar@{-->}[ur] & S }\]
        has a solution.
        \item The lifting problem 
        \[\begin{tikzcd}
            \Lambda_0[2] \ar[d] \ar[r] & X \ar[d] \\
            \Delta[2] \ar[r] \ar[ur,dashed] & S
        \end{tikzcd}\]
        % \[\xymatrix{ \Lambda_0[2] \ar[d] \ar[r] & X \ar[d] \\ \Delta[2] \ar[r] \ar@{-->}[ur] & S }\]
        has a unique solution.
    \end{itms}
    In these diagrams, the notations $\Lambda_0[n]$ and $\Delta[n]$
    represent their corresponding ordinary categories.
    Precisely, they are $\Ho(\Lambda_0[n])$ and $\Ho(\Delta[n])$.
\end{definition}

As an exercise, the reader can verify that 
this definition is equivalent to the two transport axioms given
above it.

\begin{remark}
    The uniqueness in the second axiom 
    is actually a consequence of the lifting property
    for $\Lambda_0[3]$, and thus will be replaced by the latter
    in the $\infty$-categorical definition. \varqed
\end{remark}

If we take the map $\Delta[2]\to S$ to be the constant map at $s$,
we see immediately that every morphism in the fibre $X_s$
has a left inverse, and hence $X_s$ is a groupoid.

The discussion above readily implies the following.

\begin{theorem}[Grothendieck construction]
    A left fibration between small categories $X\to S$
    gives rise to a $2$-functor $S\to\cat{Gpd}$ by taking the fibres.
    Conversely, every $2$-functor $S\to\cat{Gpd}$
    corresponds to a left fibration $X\to S$ in this manner.
\end{theorem}

\begin{proof}
    We only need to prove the converse.
    However, the construction of such a left fibration
    is done in the same way as in (\ref{eg-5-q}).
\end{proof}

Dually, a functor $p\:X\to S$ is a \term{right fibration}
if $p\op\:X\op\to S\op$ is a left fibration.
Under the Grothendieck construction,
they correspond to functors $S\op\to\cat{Gpd}$.

\subsection{For left fibrations}

Our next goal is to define Grothendieck construction
for $\infty$-categories.
As an application,
we will give a proof for the fact that 
\[\Hom_{\cat C}^{\vartriangleleft}(x,y)\simeq\Hom_{\mathfrak C\cat C}(x,y)
\simeq\Hom_{\cat C}^{\vartriangleright}(x,y)\]
for objects $x,y$ of a quasi-category $\cat C$.

\begin{definition}
    A map of simplicial sets $p\:X\to S$ is called a \term{left fibration},
    if the lifting problem 
    \[\begin{tikzcd}
        \Lambda_i[n] \ar[d] \ar[r] & X \ar[d] \\
        \Delta[n] \ar[r] \ar[ur,dashed] & S
    \end{tikzcd}\]
    % \[\xymatrix{ \Lambda_i[n] \ar[d] \ar[r] & X \ar[d] \\ \Delta[n] \ar[r] \ar@{-->}[ur] & S }\]
    has a solution for all $0\leq i<n$.

    Dually, $p$ is called a \term{right fibration}
    if the above lifting problem has a solution for all $0<i\leq n$.
\end{definition}

This definition is a natural generalisation
of the corresponding notion for ordinary categories.

\begin{proposition}
    Let $X$ be a simplicial set.
    The map $X\to*$ is a left fibration,
    if and only if $X$ is a Kan complex.
\end{proposition}

\begin{proof}
    The ``if'' part is trivial.
    For the converse, suppose that $X\to*$ is a left fibration.
    Then $X$ is a quasi-category.
    Moreover, every morphism in the ordinary category $\Ho(X)$ admits a left inverse,
    so that $\Ho(X)$ is a groupoid.
    By (\ref{thm-4-j}), $X$ must be a Kan complex.
\end{proof}

Let $S$ be a simplicial set,
and let $\cat{sSet}_{\/S}$ denote the over-category, which consists of
\begin{itms}
    \item The objects are maps $X\to S$.
    \item A morphism from a map $X\to S$ to a map $Y\to S$ is a map $X\to Y$
    that fits into a commutative triangle
    \[\begin{tikzcd}[column sep=.5em]
        X\ar[rr]\ar[dr] && Y\ar[dl] \\
        & S\rlap{\ .}
    \end{tikzcd}\]
    % \[\xymatrix @C=1em { X\ar[rr]\ar[dr] && Y\ar[dl] \\ & S\rlap{\ .} }\]
\end{itms}
Following Lurie \cite{htt}, we will define a pair of adjoint functors
\[\begin{tikzcd}[column sep=3em]
    \cat{sSet}_{\/S}\ar[r,bend left,start anchor=north east,end anchor=north west,"\operatorname{St}"]
    \ar[r,phantom,"\bot"] &
    \operatorname{Fun}_{\cat{sSet}}(\mathfrak CS,\cat{sSet})\rlap{\ ,}
    \ar[l,bend left,start anchor=south west,end anchor=south east,"\operatorname{Un}"]
\end{tikzcd}\]
% \[\xymatrix @C=3em { \cat{sSet}_{\/S}\ar@/^2ex/[r]^{\operatorname{St}} \ar@{}[r]|{\bot} & \rlap{\operatorname{Fun}_{\cat{sSet}}(\mathfrak CS,\cat{sSet})\ ,} \phantom{\cat{sSet}_{\/S}} \ar@/^2ex/[l]^{\operatorname{Un}} }\hspace{4em}\]
where $\operatorname{Fun}_{\cat{sSet}}$
denotes the (ordinary) category of simplicially enriched functors,
and we have the following.

\begin{itms}
    \item The functor $\operatorname{St}$ is called the \term{straightening functor}.
    Restricted to the left fibrations, it will give the Grothendieck construction.
    \item The functor $\operatorname{Un}$ is called the \term{unstraightening functor}.
    Restricted to the functors $\mathfrak CS\to\cat{Kan}$,
    or equivalently $S\to\mathfrak N(\cat{Kan})=:\cat{S}$,
    it will give the other direction of the construction.
    \item The adjunction $(\operatorname{St}\dashv\operatorname{Un})$
    will be a Quillen equivalence, given suitable model structures on both categories.
\end{itms}

Now, we begin the construction.

\begin{definition}
    Let $X,Y$ be simplicial sets. Their \term{join}
    is the simplicial set $X\star Y$, whose $n$ simplices are 
    \[(X\star Y)_n:=\Biggl\{(f,\sigma_0,\sigma_1)\mathrel{\Bigg|}
    \begin{array}{l}
        f\:\Delta[n]\to\Delta[1]\\
        \sigma_0\:f^{-1}(0)\to X\\
        \sigma_1\:f^{-1}(1)\to Y
    \end{array}\Biggr\},\]
    with the natural face and degeneracy maps.
\end{definition}

In other words, $X\star Y$ is obtained from $X\sqcup Y$
by adjoining all possible arrows (and higher dimensional arrows)
pointing from $X$ to $Y$. For example, we have
\[ \Delta[n]\star\Delta[m]\simeq\Delta[n+m+1]. \]

As special notations, write
\[X^\vartriangleright:=X\star\{\infty\}\quad\text{and}\quad X^\vartriangleleft:=\{\infty\}\star X,\]
where $\{\infty\}$ denotes a singleton set,
and the point $\infty$ is called the \term{cone point}.

\begin{construction}
    Let $X\to S$ be a map of simplicial sets.
    We construct the functor 
    \[\operatorname{St}_SX\:\mathfrak CS\to\cat{sSet}\]
    as follows. For $s\in S$, define 
    \[(\operatorname{St}_SX)(s):=\Hom_M(\infty,s),\]
    where the simplicial category $M$ is defined by 
    \[M:=\mathfrak C\bigl(X^\vartriangleleft\mathop{\sqcup}_XS\bigr).\]
    In other words, this construction is done through the following procedure.
    First, adjoint a point at infinity $\infty$ on the left of $X$, giving $X^\vartriangleleft$.
    Next, for each $s\in S$, crush the fibre $X_s$ to a point.
    Roughly speaking, the resulting category is $M$,
    which is a ``straightened'' version of $X^\vartriangleleft$.
    At this point, the morphism space $\Hom_M(\infty,s)$
    should still preserve information about $X_s$,
    since it was made out of all morphisms from $\infty$ to $X_s$.
    Indeed, we will see that it is weakly homotopy equivalent to $X_s$
    if $X\to S$ is a left fibration.
    \varqed
\end{construction}

\begin{construction}
    Let $F\:\mathfrak CS\to\cat{sSet}$ be a functor of simplicial categories.
    We construct a left fibration 
    \[\operatorname{Un}_SF\to S,\]
    so that $\operatorname{Un}_S$ becomes a right adjoint of $\operatorname{St}_S$.
    Namely, we define 
    \[ (\operatorname{Un}_SF)_n:=\biggl\{(\sigma,f)\biggmid\begin{array}{l}
        \sigma\:\Delta[n]\to S\\
        f\:\operatorname{St}_S\sigma\to F
    \end{array}\biggr\}, \]
    with a natural map to $S$.
    Its $n$-simplices are just ``maps from $\Delta[n]$ to $F$'',
    but precisely speaking, here $\Delta[n]$ should be replaced by its 
    Grothendieck construction,
    which is $\operatorname{St}_S\Delta[n]$.
    This motivates the above construction. \varqed
\end{construction}

By the constructions, we see that $\operatorname{St}_S$ and $\operatorname{Un}_S$
form an adjunction
\[\begin{tikzcd}[column sep=3em]
    \cat{sSet}_{\/S}\ar[r,bend left,start anchor=north east,end anchor=north west,"\operatorname{St}_S"]
    \ar[r,phantom,"\bot"] &
    \operatorname{Fun}_{\cat{sSet}}(\mathfrak CS,\cat{sSet})\rlap{\ .}
    \ar[l,bend left,start anchor=south west,end anchor=south east,"\operatorname{Un}_S"]
\end{tikzcd}\]
% \[\xymatrix @C=3em { \cat{sSet}_{\/S}\ar@/^2ex/[r]^{\operatorname{St}_S} \ar@{}[r]|{\bot} & \rlap{\operatorname{Fun}_{\cat{sSet}}(\mathfrak CS,\cat{sSet})\ ,} \phantom{\cat{sSet}_{\/S}} \ar@/^2ex/[l]^{\operatorname{Un}_S} }\hspace{4em}\]
We will sometimes omit the subscript $S$ and simply write
$\operatorname{St}$ and $\operatorname{Un}$.

\begin{example} \label{eg-5-pt}
    Let us consider the case $S=\{*\}$.
    In this case, the Grothendieck construction converts between
    fibrations onto $\{*\}$ and their fibres. We expect that the adjunction 
    \[\begin{tikzcd}[column sep=3em]
        \cat{sSet}\ar[r,bend left,"\operatorname{St}_{\{*\}}"]
        \ar[r,phantom,"\bot"] &
        \cat{sSet}
        \ar[l,bend left,"\operatorname{Un}_{\{*\}}"]
    \end{tikzcd}\]
    should be an equivalence.
    
    By definition, We have 
    \[ \operatorname{St}_{\{*\}}\Delta[n]\simeq\Hom_{\mathfrak C(\Delta[n]\/\Delta[n-1])}(0,*), \]
    where $\Delta[n-1]\hookrightarrow\Delta[n]$ as the $0$-th face $\{1,\dotsc,n\}$,
    and $\Delta[n]\/\Delta[n-1]$ denotes the simplicial set
    obtained from $\Delta[n]$ by crushing the $0$-th face $\Delta[n-1]$ to a point,
    and we denote this point by $*$.

    Let $Q^\bullet$ denote the cosimplicial simplicial set defined by 
    \[ Q^n:=\Hom_{\mathfrak C(\Delta[n]\/\Delta[n-1])}(0,*). \]
    Then the adjunction
    $(\operatorname{St}_{\{*\}}\dashv\operatorname{Un}_{\{*\}})$
    is given by $Q^\bullet$ via the construction (\ref{con-3-a}).
    With careful calculation (which we omit here; see \cite[Remark~2.2.2.6]{htt}),
    one sees that there is a homeomorphism
    \[ |Q^n|\simeq\Delta^n \]
    of topological spaces, which is compatible with the coface maps (but not the codegeneracy maps).
    It follows that for any $X\in\cat{sSet}$, there is a homeomorphism
    \[ |{\operatorname{St}_{\{*\}}X}|\simeq|X|, \]
    so that $\operatorname{St}_{\{*\}}X$ and $X$ are weakly equivalent.

    This calculation also implies that
    $\operatorname{St}_{\{*\}}$ preserves cofibrations and trivial cofibrations,
    so that $(\operatorname{St}_{\{*\}}\dashv\operatorname{Un}_{\{*\}})$
    is a Quillen equivalence, i.e., equivalence up to homotopy. \varqed
\end{example}

For a general $S\in\cat{sSet}$, we have the following result.

\begin{theorem}\label{thm-5-g}
    The category $\cat{sSet}_{\/S}$ has the \term{covariant model structure}, with
    \begin{itms}
        \item $\cat{W}$ is the class of maps $X\to Y$ such that the induced map 
        \[ X^\vartriangleleft\mathop{\sqcup}_XS\to Y^\vartriangleleft\mathop{\sqcup}_YS \]
        is a categorical equivalence.
        \item $\cat{Cof}=\{$injections$\}$.
        \item $\cat{Fib}\subset\{$left fibrations$\}$.
        \item The fibrant objects are left fibrations over $S$.
    \end{itms}
    The category $\operatorname{Fun}_{\cat{sSet}}(\mathfrak CS,\cat{sSet})$
    has the \term{projective model structure}, with
    \begin{itms}
        \item $\cat W=\{$pointwise weak equivalences$\}$.
        \item $\cat{Fib}=\{$pointwise fibrations$\}$.
        \item $\cat{Cof}$ is determined by the lifting property.
    \end{itms}
    Using these model structures, the adjunction
    \[\begin{tikzcd}[column sep=3em]
        \cat{sSet}_{\/S}\ar[r,bend left,start anchor=north east,end anchor=north west,"\operatorname{St}_S"]
        \ar[r,phantom,"\bot"] &
        \operatorname{Fun}_{\cat{sSet}}(\mathfrak CS,\cat{sSet})
        \ar[l,bend left,start anchor=south west,end anchor=south east,"\operatorname{Un}_S"]
    \end{tikzcd}\]
    % \[\xymatrix @C=3em { \cat{sSet}_{\/S}\ar@/^2ex/[r]^{\operatorname{St}_S} \ar@{}[r]|{\bot} & \rlap{\operatorname{Fun}_{\cat{sSet}}(\mathfrak CS,\cat{sSet})\ ,} \phantom{\cat{sSet}_{\/S}} \ar@/^2ex/[l]^{\operatorname{Un}_S} }\hspace{4em}\]
    is a Quillen equivalence.
\end{theorem}

See \cite[Theorem~2.2.1.2]{htt}.

\begin{corollary}[Grothendieck construction]
    The functors 
    \[\begin{tikzcd}
        \smash{\cat{LFib}_{\/S}}\vphantom{1_1}\ar[r,bend left,start anchor=north east,end anchor=north west,"R\operatorname{St}_S"] &
        \operatorname{Fun}_{\cat{sSet}}(\mathfrak CS,\cat{Kan})\ar[l,bend left,start anchor=south west,end anchor=south east,"\operatorname{Un}_S"]
    \end{tikzcd}\]
    are inverses of each other up to weak equivalences, where 
    \begin{itms}
        \item $R$ denotes the fibrant replacement functor on
        $\operatorname{Fun}_{\cat{sSet}}(\mathfrak CS,\cat{sSet})$.
        \item $\cat{LFib}_{\/S}$ is the category of left fibrations over $S$,
        as a full subcategory of $\cat{sSet}_{\/S}$.
        \item The weak equivalences on $\cat{LFib}_{\/S}$
        are fibrewise homotopy equivalences.
        \item The weak equivalences on 
        $\operatorname{Fun}_{\cat{sSet}}(\mathfrak CS,\cat{Kan})$
        are pointwise homotopy equivalences.
    \end{itms}
\end{corollary}

\begin{proof}
    Let $X\to S$ be a left fibration, and let $F\:\mathfrak CS\to\cat{Kan}$ be a functor.
    The unit and counit maps
    \[ \operatorname{Un}_SR\operatorname{St}_SX\to X\quad\text{and}\quad F\to\operatorname{St}_SQ\operatorname{Un}_SF \]
    are categorical equivalences (\ref{thm-2-z}).
    Here we may take $Q$ to be the identity functor, 
    which implies that the functors
    \[ \operatorname{Un}_SR\operatorname{St}_SX\to X\quad\text{and}\quad F\to R\operatorname{St}_S\operatorname{Un}_SF \]
    are weak equivalences.
    The statement on weak equivalences on $\cat{LFib}_{\/S}$
    will follow from (\ref{thm-5-s}) below.
\end{proof}

\begin{remark}
    Dually, for right fibrations, we have the \term{contravariant model structure}
    on $\cat{sSet}_{\/S}$, and a Quillen equivalence
    \[\begin{tikzcd}[column sep=3em]
        \cat{sSet}_{\/S}\ar[r,bend left,start anchor=north east,end anchor=north west,"\operatorname{St}_S"]
        \ar[r,phantom,"\bot"] &
        \operatorname{Fun}_{\cat{sSet}}(\mathfrak CS\op,\cat{sSet})\rlap{\ ,}
        \ar[l,bend left,start anchor=south west,end anchor=south east,"\operatorname{Un}_S"]
    \end{tikzcd}\]
    % \[\xymatrix @C=3em { \cat{sSet}_{\/S}\ar@/^2ex/[r]^{\operatorname{St}_S} \ar@{}[r]|{\bot} & \rlap{\operatorname{Fun}_{\cat{sSet}}(\mathfrak CS\op,\cat{sSet})\ ,} \phantom{\cat{sSet}_{\/S}} \ar@/^2ex/[l]^{\operatorname{Un}_S} }\hspace{4em}\]
    where $\operatorname{Fun}_{\cat{sSet}}(\mathfrak CS\op,\cat{sSet})$
    is equipped with the projective model structure. \varqed
\end{remark}

Now we can prove the most desired property of this construction,
which states that the Grothendieck construction 
gives back the fibres of a left fibration.

\begin{proposition}\label{thm-5-s}
    We have the following.
    \begin{itms}
        \item Let $X\to S$ be a left fibration.
        Then for any $s\in S$, there is a weak homotopy equivalence
        \[ (\operatorname{St}_SX)(s)\simeq X_s. \]
        \item Let $F\:\mathfrak CS\to\cat{Kan}$ be a functor.
        Then for any $s\in S$, there is a homotopy equivalence of Kan complexes
        \[ (\operatorname{Un}_SF)_s\simeq F(s). \]
    \end{itms}
\end{proposition}

\begin{proof}
    For the second part, note that by the definition of unstraightening,
    \[ (\operatorname{Un}_SF)_s \simeq \operatorname{Un}_{\{s\}}F(s) \simeq F(s), \]
    where the second equivalence uses (\ref{eg-5-pt}).

    For the first part, consider the weak equivalence
    \[ \operatorname{St} X \simto F := R\operatorname{St}X. \]
    This implies that we have a weak equivalence
    \[ X \simto \operatorname{Un} F \]
    of cofibrant-fibrant objects in $\cat{sSet}_{\/S}$.
    By (\ref{thm-2-w}), this map is a homotopy equivalence.
    In particular, we may take $X \times \Delta[1]$ and $\operatorname{Un} F \times \Delta[1]$
    as the cylinder objects,
    and a homotopy equivalence with respect to these cylinder objects
    must induce a homotopy equivalence of the fibres at each point of $S$.
    It follows that
    \[ X_s \simeq (\operatorname{Un}F)_s \]
    for any $s\in S$. By the second part, we are done.
\end{proof}

Our goal now is to use the Grothendieck construction 
to study the hom-spaces of a quasi-category.

Let $\cat C$ be a quasi-category, and let $K$ be a simplicial set.
Let $p\:K\to\cat C$ be a map of simplicial sets,
which we regard as a commutative diagram in $\cat C$.

\begin{definition}
    The \term{over-category} $\cat C_{\/K}$ is a quasi-category,
    with $n$-simplices
    \[ (\cat C_{\/K})_n:=\bigl\{\sigma\:\Delta[n]\star K\to\cat C\bigmid\sigma|_K=p\bigr\}. \]
    Dually, the \term{under-category} $\cat C_{K\/}$ is a quasi-category,
    with $n$-simplices
    \[ (\cat C_{K\/})_n:=\bigl\{\sigma\:K\star\Delta[n]\to\cat C\bigmid\sigma|_K=p\bigr\}. \]
\end{definition}

The following is an immediate consequence of (\ref{thm-5-s}).

\begin{corollary}
    Let $\cat C$ be a quasi-category, and let $x,y\in\cat C$. Then 
    \[ \Hom_{\cat C}^\vartriangleleft(x,y)\simeq(\operatorname{St}_{\cat C}\cat C_{x\/})(y).
    \qedhere \] \qed
\end{corollary}

Moreover, we have the following.

\begin{proposition}
    Let $\cat C$ be a quasi-category, and let $x,y\in\cat C$. Then 
    \[ (\operatorname{St}_{\cat C}\cat C_{x\/})(y)\simeq\Hom_{\mathfrak C\cat C}(x,y). \]
\end{proposition}

\begin{proof}
    By the definition of straightening, we see that 
    \[ \Hom_{\mathfrak C\cat C}(x,y) \simeq (\operatorname{St}_{\cat{C}}\{x\})(y). \]
    Therefore, it suffices to show that the map of functors 
    \[ \operatorname{St}_{\cat{C}}\{x\} \to \operatorname{St}_{\cat{C}}\cat{C}_{x\/} \]
    is a weak equivalence. In fact, we will show that it is a trivial cofibration,
    which reduces to showing that the inclusion
    \[ \{x\} \hookrightarrow \cat{C}_{x\/} \]
    is a trivial cofibration in $\cat{sSet}_{\/\cat{C}}$.
    Thus we only need to show that this map has the left lifting property against left fibrations.
    This is because this map is a retract of the map 
    \[ \{\infty\} \hookrightarrow (\cat{C}_{x\/})^\vartriangleleft, \]
    which has the required left lifting property,
    since it is obtained by attaching simpices along left horns.
\end{proof}

Combining these two equivalences, we see that
\[\Hom_{\cat C}^\vartriangleleft(x,y)\simeq\Hom_{\mathfrak C\cat C}(x,y).\]
Together with a dual argument, involving right fibrations,
we have shown that 

\begin{theorem} \label{thm-5-h}
    Let $\cat C$ be a quasi-category, and let $x,y\in\cat C$. Then 
    \[\Hom_{\cat C}^\vartriangleleft(x,y)\simeq\Hom_{\mathfrak C\cat C}(x,y)
    \simeq\Hom_{\cat C}^\vartriangleright(x,y). \qedhere \] \restoreqed
\end{theorem}

This is our first application of the Grothendieck construction.

\subsection{For cocartesian fibrations}

Recall that we have a table of Grothendieck constructions.

\begin{center}
    \begin{tabular}{ccc}
        \textbf{Fibration} & & \textbf{Functor} \\ \hline
        left fibration $X\to S$ & $\Longleftrightarrow$ & $S\to\cat{Gpd}$ \\
        right fibration $X\to S$ & $\Longleftrightarrow$ & $S\op\to\cat{Gpd}$ \\
        cocartesian fibration $X\to S$ & $\Longleftrightarrow$ & $S\to\cat{Cat}$ \\
        cartesian fibration $X\to S$ & $\Longleftrightarrow$ & $S\op\to\cat{Cat}$ \\
    \end{tabular}
\end{center}

Now, we sketch the construction for cocartesian and cartesian fibrations.

Let $p\:X\to S$ be a functor between ordinary categories.
A morphism $f\:x\to y$ in $X$ is said to be \term{$p$-cocartesian},
if we have a ``cocartesian square''
\[\begin{tikzcd}
    x\ar[d,dotted,dash]\ar[r,"f"] & y\ar[d,dotted,dash] \\
    p(x)\ar[r,"p(f)"]\ar[ur,phantom,end anchor=south west,very near end,"\urcorner"] & p(y)\rlap{\ .}
\end{tikzcd}\]
Precisely speaking, the map $x\to y$ has the following universal property.
For any commutative diagram without the dashed arrow,
\[\begin{tikzcd}[row sep=1em]
    && z\ar[dd,dotted,dash] \\
    x\ar[dd,dotted,dash]\ar[r,"f"']\ar[urr,"g"]
    & y\ar[dd,dotted,dash]\ar[ur,dashed,end anchor=south west,"\exists!\ \tilde h"' inner sep=0pt] \\
    && p(z) \\
    p(x)\ar[r,"p(f)"']\ar[urr,"p(g)"]
    & p(y)\rlap{\ ,} \ar[ur,"h"']
\end{tikzcd}\]
there exists a unique morphism $\tilde h\:y\to z$,
such that $\tilde hf=g$ and $p(\tilde h)=h$.

We reformulate this definition as follows.

\begin{definition}
    Let $p\:X\to S$ be a functor between ordinary categories.
    A morphism $f\:x\to y$ in $X$ is said to be \term{$p$-cocartesian},
    if the lifting problem 
    \[\begin{tikzcd}
        \Lambda_0[2] \ar[d] \ar[r] & X \ar[d] \\
        \Delta[2] \ar[r] \ar[ur,dashed] & S
    \end{tikzcd}\]
    % \[\xymatrix{ \Lambda_0[2] \ar[d] \ar[r] & X \ar[d] \\ \Delta[2] \ar[r] \ar@{-->}[ur] & S }\]
    has a unique solution, whenever the edge $[0,1]$ in $\Lambda_0[2]$ is sent to $f$ in $X$.
\end{definition}

This definition generalises naturally to quasi-categories.

\begin{definition}
    Let $p\:X\to S$ be a functor between quasi-categories.
    A morphism $f\:x\to y$ in $X$ is said to be \term{$p$-cocartesian},
    if the lifting problem 
    \[\begin{tikzcd}
        \Lambda_0[n] \ar[d] \ar[r] & X \ar[d] \\
        \Delta[n] \ar[r] \ar[ur,dashed] & S
    \end{tikzcd}\]
    % \[\xymatrix{ \Lambda_0[n] \ar[d] \ar[r] & X \ar[d] \\ \Delta[n] \ar[r] \ar@{-->}[ur] & S }\]
    has a solution whenever the edge $[0,1]$ in $\Lambda_0[n]$ is sent to $f$ in $X$.
\end{definition}

As before, the uniqueness condition is dropped,
as it is replaced by higher lifting properties.

\begin{definition}
    Let $p\:X\to S$ be a functor between ordinary categories.
    Then $p$ is called a \term{cocartesian fibration}, if $X$ admits ``pushouts'' from $S$:
    \[\begin{tikzcd}
        x\ar[d,dotted,dash] \\
        s \ar[r] & s'
    \end{tikzcd}\quad\Longrightarrow\quad\begin{tikzcd}
        x\ar[r,dashed]\ar[d,dotted,dash] & x'\ar[d,dotted,dash] \\
        s \ar[r]\ar[ur,phantom,end anchor=south west,very near end,"\urcorner"] & s'\rlap{\ .}
    \end{tikzcd}\]
    Precisely speaking, for any
    $x\in X$ and any morphism $\alpha\:s\to s'$ in $S$,
    where $s:=p(x)$,
    there exists a $p$-cocartesian morphism $\tilde\alpha\:x\to x'$ in $X$,
    such that $p(\tilde\alpha)=\alpha$.
\end{definition}

For quasi-categories, we have an extra requirement.

\begin{definition}
    Let $p\:X\to S$ be a functor between quasi-categories.
    Then $p$ is called a \term{cocartesian fibration}, if
    \begin{itms}
        \item The lifting problem 
        \[\begin{tikzcd}
            \Lambda_i[n] \ar[d] \ar[r] & X \ar[d] \\
            \Delta[n] \ar[r] \ar[ur,dashed] & S
        \end{tikzcd}\]
        % \[\xymatrix{ \Lambda_i[n] \ar[d] \ar[r] & X \ar[d] \\ \Delta[n] \ar[r] \ar@{-->}[ur] & S }\]
        has a solution if $0<i<n$. That is, $p$ is an \term{inner fibration}.
        \item $X$ admits ``pushouts'' from $S$:
        \[\begin{tikzcd}
            x\ar[d,dotted,dash] \\
            s \ar[r] & s'
        \end{tikzcd}\quad\Longrightarrow\quad\begin{tikzcd}
            x\ar[r,dashed]\ar[d,dotted,dash] & x'\ar[d,dotted,dash] \\
            s \ar[r]\ar[ur,phantom,end anchor=south west,very near end,"\urcorner"] & s'\rlap{\ .}
        \end{tikzcd}\]
    \end{itms}
\end{definition}

The first condition is a general requirement for fibration-like maps of quasi-categories,
and it is satisfied by any functor between ordinary categories.

\begin{proposition}
    A left fibration $X\to S$ is equivalently a cocartesian fibration
    such that every edge of $X$ is cocartesian. \qed
\end{proposition}

Cocartesian fibrations provide an obvious way to 
define transport functors between the fibres,
which implies the following theorem.

\begin{theorem}[Grothendieck construction]
    A cocartesian fibration $p\:X\to S$ of ordinary categories
    gives rise to a $2$-functor $S\to\cat{Cat}$ by taking the fibres.
    Conversely, every $2$-functor $S\to\cat{Cat}$
    corresponds to a cocartesian fibration $X\to S$ in this manner. \qed
\end{theorem}

For quasi-categories, one needs a ``marked'' version
of the straightening and unstraightening construction.
We define $\cat{sSet}^+$ to be the category of simplicial sets 
with some marked edges, such that all degenerate edges are marked.
The morphisms in $\cat{sSet}^+$ are required to send marked edges to marked edges.

For a simplicial set $S$, let $S^\sharp$ denote 
the marked version of $S$ in which all edges are marked,
and let $S^\flat$ denote the marked version of $S$
in which only the degenerate edges are marked.
We denote $(\cat{sSet}^+)_{\/{S^\sharp}}$ by $\cat{sSet}^+_{\/S}$.

\begin{theorem}
    There exists a Quillen equivalence
    \[\begin{tikzcd}[column sep=3em]
        \cat{sSet}^+_{\/S}\ar[r,bend left,start anchor=north east,end anchor=north west,"\operatorname{St}_S^+"]
        \ar[r,phantom,"\bot"] &
        \vphantom{\cat{sSet}^+_{\/S}}
        \operatorname{Fun}_{\cat{sSet}}(\mathfrak CS,\cat{sSet}^+)\rlap{\ ,}
        \ar[l,bend left,start anchor=south west,end anchor=south east,"\operatorname{Un}_S^+"]
    \end{tikzcd}\]
    given suitable model structures on both categories.
    Moreover, 
    \begin{itms}
        \item $\operatorname{Un}_S^+$ sends functors to
        $\cat{QsCat}\simeq(\cat{sSet}^+)_{\mathrm{cf}}$
        to cocartesian fibrations.
        \item $\operatorname{St}_S^+$ gives back the fibres
        for cocartesian fibrations,
        up to a categorical equivalence.
    \end{itms}
\end{theorem}

See \cite[Theorem~3.2.0.1]{htt}.

\begin{corollary}[Grothendieck construction]
    The functors 
    \[\begin{tikzcd}
        \smash{\cat{CocFib}_{\/S}}\vphantom{1_1}\ar[r,bend left,start anchor=north east,end anchor=north west,"R\operatorname{St}_S^+"] &
        \operatorname{Fun}_{\cat{sSet}}(\mathfrak CS,\cat{QsCat})\ar[l,bend left,start anchor=south west,end anchor=south east,"\operatorname{Un}_S^+"]
    \end{tikzcd}\]
    are inverses of each other up to weak equivalences, where 
    \begin{itms}
        \item $R$ denotes the fibrant replacement functor on
        $\operatorname{Fun}_{\cat{sSet}}(\mathfrak CS,\cat{sSet}^+)$.
        \item $\cat{CocFib}_{\/S}$ is the category of cocartesian fibrations over $S$.
        \item The weak equivalences on $\cat{CocFib}_{\/S}$
        are fibrewise categorical equivalences.
        \item The weak equivalences on 
        $\operatorname{Fun}_{\cat{sSet}}(\mathfrak CS,\cat{QsCat})$
        are pointwise categorical equivalences. \qed
    \end{itms}
\end{corollary}

\begin{example}
    Let $S=\cat{Cat}_\infty$ be the category of quasi-categories,
    and let $F:=(-)^\flat\in\operatorname{Fun}(S,\cat{sSet}^+)$
    be the inclusion functor.
    The cocartesian fibration 
    \[ \operatorname{Un}_S^+F=:\mathscr Z\to\cat{Cat}_\infty \]
    is a \term{universal fibration}, in that every cocartesian fibration 
    is equivalent to its pullback.
    Namely, let $X\to T$ be a cocartesian fibration.
    Then there exists a \term{classifying map}
    \[ f:=\operatorname{St}_T^+X\:T\to\cat{Cat}_\infty, \]
    so that 
    \[ X\simeq f^*\mathscr Z. \]
    Roughly speaking, this is because
    $X$ and $f^*\mathscr Z$ both are cocartesian fibrations over $S$,
    and they have the same fibres.
    \varqed
\end{example}
