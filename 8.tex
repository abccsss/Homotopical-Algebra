The following two sections will be focused on 
the study of homological algebra. 
Our goal is to establish homological algebra 
as a special case of homotopical algebra.
We will construct and study the $\infty$-category of chain complexes,
and the $\infty$-categorical version of the derived category.

In this section, we start from the Dold--Kan correspondence,
which is a classical result in homological algebra
that establishes an equivalence
between chain complexes in an abelian category~$\cat{A}$
and simplicial objects in~$\cat{A}$.
It gives an equivalence of (ordinary) categories
\[ \cat{Ch}(\cat{A})_{\geq0}\xrightarrow[\textstyle\widetilde{\phantom{......}}]{\operatorname{DK}}
\operatorname{Fun}(\bfDelta\op,\cat{A}), \]
where $\cat{Ch}(\cat{A})_{\geq0}$ denotes the full subcategory of~$\cat{Ch}(\cat{A})$
consisting of chain complexes that terminate at the $0$th term.
This equivalence will
relate the homotopy theory of chain complexes
to the homotopy theory of simplicial sets.

\subsection{t-structures}

\begin{definition}
    Let $\cat{C}$ be a triangulated category.
    A \term{t-structure} on~$\cat{C}$ consists of two full subcategories
    \[ \cat{C}^{\leq0},\ \cat{C}^{\geq0}\subset\cat C, \]
    satisfying the following axioms: if we denote
    \[ \cat{C}^{\leq n}:=\cat{C}^{\leq0}[n]\quad\text{and}\quad
    \cat{C}^{\geq n}:=\cat{C}^{\geq0}[n] \]
    for $n\in\mathbb Z$, then
    \begin{itms}
        \item $\cat{C}^{\leq0}$ and $\cat{C}^{\geq0}$
        are stable under isomorphisms.
        \item $\cat{C}^{\leq0}\subset\cat{C}^{\leq1}$ and $\cat{C}^{\geq1}\subset\cat{C}^{\geq0}$.
        \item If $X\in\cat{C}^{\leq0}$ and $Y\in\cat{C}^{\geq1}$, then
        $\Hom_{\cat{C}}(X,Y)=0$.
        \item For every $X\in\cat{C}$, there exists a distinguished triangle 
        \[X^{\leq0}\to X\to X^{\geq1},\]
        such that $X^{\leq0}\in\cat{C}^{\leq0}$ and $X^{\geq1}\in\cat{C}^{\geq1}$.
    \end{itms}
\end{definition}

For example, if $\cat{C}=\cat{Ch}(\cat{A})$ is the category of
cochain complexes in an abelian category~$\cat{A}$,
then we may take $\cat{C}^{\leq0}$ to be those cochain complexes~$X$
with $X_n=0$ for all $n>0$. Similarly we can define $\cat{C}^{\geq0}$.
Then the first three axioms are immediate, and for the fourth one, we may take 
\[ X^{\leq0}:=\tau^{\leq0}X\quad\text{and}\quad X^{\geq1}:=\tau^{\geq1}X, \]
where $\tau^{\leq n}$ and $\tau^{\geq n}$ are the truncation functors, defined by
\[ \begin{aligned}
    \tau^{\leq n}X&:=\bigl(\cdots\to X^{n-2}\xrightarrow{d_{n-2}}X^{n-1}\xrightarrow{d_{n-1}}
    \operatorname{ker}d_{n}\to0\to\cdots\bigr), \\
    \tau^{\geq n}X&:=\bigl(\cdots\to0\to\operatorname{coker}d_{n-1}\xrightarrow{d_n}X^{n+1}
    \xrightarrow{d_{n+1}}X^{n+2}\to\cdots\bigr).
\end{aligned} \]

\begin{remark}
    If $X\in\cat{C}^{\leq0}$ and $Y\in\cat{C}^{\geq1}$, we require that
    $\Hom_{\cat{C}}(X,Y)=0$, but not $\Hom_{\cat{C}}(Y,X)=0$,
    and here is a justification.
    Let us look at $\cat{C}=\cat{Ch}(\cat{A})$ as an example.
    In this case, although the only chain map from $Y$ to $X$ is zero,
    there may exist nonzero chain homotopies and higher homotopies. \varqed
\end{remark}

\begin{notation}
    Sometimes it is more convenient to use homological indexing instead of cohomological indexing.
    Therefore, we introduce the following notation.
    For a triangulated category~$\cat{C}$ with a t-structure, we denote 
    \[ \cat{C}_{\geq n}:=\cat{C}^{\leq-n}\quad\text{and}\quad\cat{C}_{\leq n}:=\cat{C}^{\geq-n}. \]
    Using this notation, we have $\cat{C}_{\geq n}=\cat{C}_{\geq0}[-n]$, etc. \varqed
\end{notation}

\begin{definition}
    Let $\cat{C}$ be a stable $\infty$-category.
    A \term{t-structure} on~$\cat{C}$ is a t-structure on
    the triangulated category~$\Ho(\cat{C})$.
\end{definition}

An important property of a t-structure is that 
it establishes $\cat{C}^{\geq n}$ as a \emph{localisation} of~$\cat{C}$,
in the sense that it is equivalent to $\cat{C}[\cat{W}^{-1}]$ 
for some collection~$\cat{W}$ of arrows in~$\cat{C}$.

\begin{definition}
    A functor $f\:\cat{C}\to\cat{D}$ between $\infty$-categories
    is called a \term{localisation functor}, if 
    it has a fully faithful right adjoint.
\end{definition}

\begin{remark}
    This definition of localisation
    is narrower than what we called localisations before,
    i.e.\ functors obtained by inverting some of the arrows.
    In fact, if $f$ is a localisation functor in this new sense,
    then $f$ is obtained by inverting all morphisms in~$\cat{C}$
    which are sent to equivalences in~$\cat{D}$.
    For a proof of this fact, see \cite[Proposition~5.2.7.12]{htt}. \varqed
\end{remark}

\begin{theorem}
    Let $\cat{C}$ be a stable $\infty$-category with a t-structure,
    and let $n\in\mathbb Z$. Then there exists a localisation functor
    \[ \tau^{\geq n}\:\cat{C}\to\cat{C}^{\geq n}, \]
    called the \term{truncation functor}, which is left adjoint
    to the inclusion functor.
\end{theorem}

\begin{proof}
    \nyw
\end{proof}

As a corollary, the subcategory $\cat{C}^{\geq n}$ of $\cat{C}$
is stable under limits which exist in $\cat{C}$.
Dually, $\cat{C}^{\leq n}$ is stable under colimits.

\begin{definition}
    Let $\cat{C}$ be a stable $\infty$-category with a t-structure.
    The \term{heart} of $\cat{C}$ is defined to be 
    \[ \cat{C}^\heartsuit:=\cat{C}^{\geq0}\cap\cat{C}^{\leq0}\subset\cat{C}. \]
    We define a family of functors
    \[ \pi_n\:\cat{C}\to\cat{C}^\heartsuit \]
    by $\pi_0:=\tau^{\geq0}\tau^{\leq0}\simeq\tau^{\leq0}\tau^{\geq0}$,
    and $\pi_n:=\pi_0\circ[-n]$.
\end{definition}

For a proof of $\tau^{\geq0}\tau^{\leq0}\simeq\tau^{\leq0}\tau^{\geq0}$,
see \cite[Proposition~1.2.1.10]{ha}.

For example, if $\cat{C}=\cat{Ch}(\cat{A})$,
then $\cat{C}^\heartsuit\simeq\cat{A}$,
and $\pi_n$ takes the $(-n)$-th cohomology group,
i.e.\ the $n$-th homology group, of a chain complex.

\subsection{Dold--Kan correspondence}

Let $\cat{A}$ be an abelian category.
In this section, we will define a functor 
\[\operatorname{DK}\:\cat{Ch}(\cat{A})_{\geq0} \to \operatorname{Fun}(\bfDelta\op,\cat{A}),\]
and we will prove that this functor 
is an equivalence of categories.

\begin{construction}
    Let $\cat{A}$ be an abelian category,
    and let $X_\bullet\in\cat{Ch}(\cat{A})$.
    The simplicial object 
    \[\operatorname{DK}_\bullet(X)\in\operatorname{Fun}(\bfDelta\op,\cat{A})\]
    is represented by the following picture.
    
    \begin{center}
        \begin{tikzpicture}[>={Straight Barb}, line width=.5pt]
            \node (P0) at (0,0) {};
            \node (P1) at (3.5,-1) {};
            \node (P2) at (5,.2) {};
            \node (P3) at (2.8,3.7) {};

            \node (X0) at (-.3,-.2) {$X_0$};
            \node (X1) at (1.4,-.7) {$X_1$};
            \node (X2) at (2.8,-.4) {\smash{$X_2$}};
            \node (X3) at (2.6,1.1) {$X_3$};
            \node (X4) at (1.6,1.4) {$\cdots$};
            \node at (3.5,-1.3) {$0$};
            \node at (5.3,.2) {$0$};
            \node at (2.8,4) {$0$};
            \node at (4.4,-.6) {$0$};
            \node at (2.2,-.1) {$0$};
            \node at (1.2,2) {$0$};
            \node at (3.3,1.8) {$0$};
            \node at (4.2,2) {$0$};

            \fill (P0) ellipse (.05);
            \fill (P1) ellipse (.05);
            \fill (P2) ellipse (.05);
            \fill (P3) ellipse (.05);

            \draw [->] (X2) to[bend left=40] (X1);

            \draw [line width=5pt, white, shorten <=10pt, shorten >=10pt] (P1) -- (P0);
            \draw [->, shorten <=5pt, shorten >=3pt] (P1) -- (P0);
            \draw [->, shorten <=5pt, shorten >=3pt] (P2) -- (P0);
            \draw [->, shorten <=5pt, shorten >=3pt] (P2) -- (P1);
            \draw [->, shorten <=5pt, shorten >=3pt] (P3) -- (P0);
            \draw [line width=5pt, white, shorten <=5pt, shorten >=5pt] (P3) -- (P1);
            \draw [->, shorten <=5pt, shorten >=3pt] (P3) -- (P1);
            \draw [->, shorten <=5pt, shorten >=3pt] (P3) -- (P2);

            \draw [->] (X1) to[bend left=40] (X0);
            \draw [line width=5pt, white, shorten >=5pt] (X3) to[bend left=20] (X2);
            \draw [->, shorten >=5pt] (X3) to[bend left=20] (X2);
            \draw [->, dashed] (X4) to[bend left=40] (X3);

            \node at (.6,-1.1) {$\scriptstyle d_0=\partial_1$};
            \node at (2.2,-1.1) {$\scriptstyle d_0=\partial_2$};
            \node at (2.45,.4) {$\scriptstyle d_0=\partial_3$};
        \end{tikzpicture}
    \end{center}

    A formal definition may go as follows. 
    \begin{itms}
        \item For every $n\geq0$, the space of $n$-simplices is given by
        \[\operatorname{DK}_n(X):=\bigoplus_{\substack{\alpha\:[n]\to[k]\\\text{surjective}}}X_k.\]
        \item For every morphism $\beta\:[n']\to[n]$ in $\bfDelta$,
        the induced map
        \[\beta^*\:\operatorname{DK}_{n}(X)\to\operatorname{DK}_{n'}(X)\]
        is given by the maps 
        \[f_{\alpha,\alpha'}\:X_k\to X_{k'}\]
        for surjective maps $\alpha\:[n]\to[k]$ and $\alpha'\:[n']\to[k']$ in $\bfDelta$,
        where 
        \[f_{\alpha,\alpha'}:=\begin{cases}
            1_{X_k}, & \text{if }k'=k\text{ and }
            \begin{tikzcd}[cramped, sep=.9em]
                \scriptstyle[n']\ar[r]\ar[d] & \scriptstyle[n]\ar[d] \\{}
                \scriptstyle[k']\ar[r] & \scriptstyle[k]
            \end{tikzcd}\text{ commutes},\\[15pt]
            \partial_k, & \text{if }k'=k-1\text{ and }
            \begin{tikzcd}[cramped, sep=.9em]
                \scriptstyle[n']\ar[r]\ar[d] & \scriptstyle[n]\ar[d] \\{}
                \scriptstyle[k']\ar[r,"d^0"] & \scriptstyle[k]
            \end{tikzcd}\text{ commutes},\\
            0, & \text{otherwise},
        \end{cases}\]
        where $\partial_k\:X_k\to X_{k-1}$ denotes the differential,
        and $d^0\:[k-1]\to[k]$ is the map sending $i\in[k-1]$ to $i+1\in[k]$. \varqed
    \end{itms}
\end{construction}

\tbc

\subsection{For \texorpdfstring{$\infty$}{∞}-categories}

\nyw
