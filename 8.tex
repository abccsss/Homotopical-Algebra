The following two sections will be focused on 
the study of homological algebra. 
Our goal is to establish homological algebra 
as a special case of homotopical algebra.
We will construct and study the $\infty$-category of chain complexes,
and the $\infty$-categorical version of the derived category.

In this section, we start from the Dold--Kan correspondence,
which is a classical result in homological algebra
that establishes an equivalence
between chain complexes in an abelian category~$\cat{A}$
and simplicial objects in~$\cat{A}$.
It gives an equivalence of (ordinary) categories
\[ \cat{Ch}(\cat{A})_{\geq0}\xrightarrow[\textstyle\widetilde{\phantom{......}}]{\operatorname{DK}}
\operatorname{Fun}(\bfDelta\op,\cat{A}), \]
where $\cat{Ch}(\cat{A})_{\geq0}$ denotes the full subcategory of~$\cat{Ch}(\cat{A})$
consisting of chain complexes that terminate at the $0$th term.
This equivalence will
relate the homotopy theory of chain complexes
to the homotopy theory of simplicial sets.

\subsection{t-structures}

\begin{definition}
    Let $\cat{C}$ be a triangulated category.
    A \term{t-structure} on~$\cat{C}$ consists of two full subcategories
    \[ \cat{C}^{\leq0},\ \cat{C}^{\geq0}\subset\cat C, \]
    satisfying the following axioms: if we denote
    \[ \cat{C}^{\leq n}:=\cat{C}^{\leq0}[n]\quad\text{and}\quad
    \cat{C}^{\geq n}:=\cat{C}^{\geq0}[n] \]
    for $n\in\mathbb Z$, then
    \begin{itms}
        \item $\cat{C}^{\leq0}$ and $\cat{C}^{\geq0}$
        are stable under isomorphisms.
        \item $\cat{C}^{\leq0}\subset\cat{C}^{\leq1}$ and $\cat{C}^{\geq1}\subset\cat{C}^{\geq0}$.
        \item If $X\in\cat{C}^{\leq0}$ and $Y\in\cat{C}^{\geq1}$, then
        $\Hom_{\cat{C}}(X,Y)=0$.
        \item For every $X\in\cat{C}$, there exists a distinguished triangle 
        \[X^{\leq0}\to X\to X^{\geq1},\]
        such that $X^{\leq0}\in\cat{C}^{\leq0}$ and $X^{\geq1}\in\cat{C}^{\geq1}$.
    \end{itms}
\end{definition}

For example, if $\cat{C}=\cat{Ch}(\cat{A})$ is the category of
cochain complexes in an abelian category~$\cat{A}$,
then we may take $\cat{C}^{\leq0}$ to be those cochain complexes~$X$
with $X_n=0$ for all $n>0$. Similarly we can define $\cat{C}^{\geq0}$.
Then the first three axioms are immediate, and for the fourth one, we may take 
\[ X^{\leq0}:=\tau^{\leq0}X\quad\text{and}\quad X^{\geq1}:=\tau^{\geq1}X, \]
where $\tau^{\leq n}$ and $\tau^{\geq n}$ are the truncation functors, defined by
\[ \begin{aligned}
    \tau^{\leq n}X&:=\bigl(\cdots\to X^{n-2}\xrightarrow{d_{n-2}}X^{n-1}\xrightarrow{d_{n-1}}
    \operatorname{ker}d_{n}\to0\to\cdots\bigr), \\
    \tau^{\geq n}X&:=\bigl(\cdots\to0\to\operatorname{coker}d_{n-1}\xrightarrow{d_n}X^{n+1}
    \xrightarrow{d_{n+1}}X^{n+2}\to\cdots\bigr).
\end{aligned} \]

\begin{remark}
    If $X\in\cat{C}^{\leq0}$ and $Y\in\cat{C}^{\geq1}$, we require that
    $\Hom_{\cat{C}}(X,Y)=0$, but not $\Hom_{\cat{C}}(Y,X)=0$,
    and here is a justification.
    Let us look at $\cat{C}=\cat{Ch}(\cat{A})$ as an example.
    In this case, although the only chain map from $Y$ to $X$ is zero,
    there may exist nonzero chain homotopies and higher homotopies. \varqed
\end{remark}

\begin{notation}
    Sometimes it is more convenient to use homological indexing instead of cohomological indexing.
    Therefore, we introduce the following notation.
    For a triangulated category $\cat{C}$ with a t-structure, we denote 
    \[ \cat{C}_{\geq n}:=\cat{C}^{\leq-n}\quad\text{and}\quad\cat{C}_{\leq n}:=\cat{C}^{\geq-n}. \]
    Using this notation, we have $\cat{C}_{\geq n}=\cat{C}_{\geq0}[-n]$, etc. \varqed
\end{notation}

\begin{definition}
    Let $\cat{C}$ be a stable $\infty$-category.
    A \term{t-structure} on $\cat{C}$ is a t-structure on
    the triangulated category $\Ho(\cat{C})$.
\end{definition}

\tbc

\subsection{Dold--Kan correspondence}

\nyw

\subsection{For \texorpdfstring{$\infty$}{∞}-categories}

\nyw
